\documentclass[preprint]{sigplanconf}

%include polycode.fmt

\usepackage{amsmath}
\usepackage{mathpartir}
\usepackage{multicol}
\usepackage{color}
\usepackage{verbatim}

\begin{document}


\conferenceinfo{Haskell Symposium '11}{22 September, Tokyo.} 
\copyrightyear{2011} 
\copyrightdata{[to be supplied]} 

%\titlebanner{Unpublished draft}        % These are ignored unless
%\preprintfooter{short description of paper}   % 'preprint' option specified.

\title{Extending monads with pattern matching}
%\subtitle{(Working title)}

\authorinfo{Tomas Petricek \and Alan Mycroft}
           {University of Cambridge}
           {\{tomas.petricek, am\}@@cl.cam.ac.uk}

\authorinfo{Don Syme}
           {Microsoft Research Cambridge}
           {don.syme@@microsoft.com}

\maketitle

\begin{abstract}
Sequencing of effectful computations can be neatly captured using monads and elegantly written using 
@do@ notation. In practice such monads often allow additional ways of composing computations, 
which have to be written explicitly using combinators.

We identify joinads, an abstract notion of computation that is stronger than monads and captures
many such ad-hoc extensions. In particular, joinads are monads with three additional operations: 
one of type |m a -> m b -> m (a, b)| captures various forms of \textit{parallel composition}, 
one of type |m a -> m a -> m a| that is inspired by \textit{choice} and one of type |m a -> m (m a)| 
that captures \textit{aliasing} of computations. Algebraically, the first two operations form a 
near-semiring with commutative multiplication.

We introduce @docase@ notation that can be viewed as a monadic version of @case@. Joinad laws
make it possible to prove various syntactic equivalences of programs written using @docase@ 
that are analogous to equivalences about @case@. Examples of joinads that benefit from the notation
include speculative parallelism, waiting for a combination of user interface events, but also 
encoding of validation rules using the intersection of parsers. 
\end{abstract}


% Sequential monadic code can be elegantly written using the @do@ notation, but many monadic 
% computations are not strictly sequential. Other forms of composition have to be written explicitly 
% using combinators, which makes code less readable and the results are difficult to use in other 
% languages. This is becoming increasingly important as monads are more often used for tasks such as 
% parallel, concurrent and reactive programming.

% We introduce docase notation that enables pattern matching on monadic values. The notation can 
% be used with any monad that provides two additional combinators. \textit{Parallel composition} of 
% type |m a -> m b -> m (a, b)| may run the two arguments in parallel and \textit{choice} of type 
% |m a -> m a -> m a| may choose the first available value. The actual meaning differs for each monad, 
% but we derive laws that guarantee an intuitive semantics similar to case. We use the notation 
% to program speculative parallelism, encode waiting for a combination of user interface events, but 
% also for encoding of validation rules using the intersection of parsers. 



\category{D.3.3}{Language Constructs and Features}{Control structures}
\category{F.1.2}{Models of Computation}{Parallelism and concurrency}

\terms
Languages, Theory

% \keywords
% keyword1, keyword2

% --------------------------------------------------------------------------------------------------
%format ^ = " "
%format ^^ = "\;"
%format instancesem_ = "\mathbf{instance_{sem}}$ \ $"
%format docase_ = "\mathbf{docase}"
%format mcase_ = "\mathbf{mcase}"
%format inf_ = "\infty"
%format forall_ = "\forall"
%format bot_ = "\bot"
%format sp_  = "$ \ $"
%format smp_  = "$\,$"
%format ssp_  = "$ \qquad \quad $"
%format lsp_  = "\\[-17pt]"
%format neq_ = "\neq"
%format morelse_ = "\oplus"
%format mplus_ = "\oplus"
%format pi_ = "\pi"
%format mzip_ = "\otimes"
%format pfail_ = "\bullet"
%format and_ = "\mathbf{and}"
%format star_ = "\star"
%format diamond_ = "\diamond"
%format sub(a) = "_{" a "}"
%format sup(a) = "^{" a "}"
%format sm(a) = "\mathopen{\llbracket} " a "\mathclose{\rrbracket} "
%format tsm(a) = "\mathopen{\langle} " a "\mathclose{\rangle} "

\newcommand{\ident}[1]{{\normalfont\sffamily #1}}
\newcommand{\code}[1]{{\normalfont\ttfamily #1}}
\newcommand{\mzip}[0]{\otimes}
\newcommand{\morelse}[0]{\oplus}
\newcommand{\mzero}[0]{\text{mzero}}
\newcommand{\map}[0]{\text{map}}
\newcommand{\mduplicate}[0]{\text{mdup}}
\newcommand{\mplus}[0]{\oplus}
\newcommand{\fsharp}[0]{F$^\#$}
\newcommand{\todo}[1]{\textcolor{red}{[TODO] #1}}

% ==================================================================================================

\section{Introduction}
Monads are traditionally used for embedding of sequential computations into lazy functional code, 
but many recent uses go well beyond sequencing of state or computations. Monads have been used for 
the exact opposite---to explicitly specify parallelism. This is done by taking a core sequential 
monad and providing additional combinators that increase the expressive power beyond sequencing.

Monads for concurrent \cite{poorman} and parallel programming \cite{parmonad} support forking a 
computation and synchronizing computations \cite{chp-monad}. A monad for user-interface programming 
includes combinators for merging events from various sources \cite{imperative-streams}. These 
ad-hoc extensions are extremely useful, but they are not uniform. Developers have to understand 
different combinators for every computation and they lose the syntactic support provided by @do@ notation.

This paper discusses \textit{joinads}---an abstract notion of computations that extends monads.
Joinads capture a pattern that appears in many monadic computations with an additional expressive 
power that goes beyond sequencing. We presented an earlier form of joinads in \fsharp \cite{joinads}. 
This paper makes several novel findings, but our first contribution to Haskell is similar to the 
earlier work in \fsharp:

\begin{itemize}
\item  We show how to add language support for important kinds of computations, including computations 
  for parallel, concurrent and reactive programming. This is done via a lightweight, reusable language 
  extension that builds on core functional concepts such as pattern matching.
\end{itemize}
This paper simplifies the concept of joinad and requires that every joinad is also a monad (just like
group is also a monoid), which allows us to build solid theoretical foundations. In the context of 
Haskell, we can also combine several ideas that already disconnectedly exist. The specific new 
contributions of this paper are:

\begin{itemize}
\item We present @docase@ notation for Haskell (Sections \ref{sec:motivation}, \ref{sec:extension}) 
  that allows programming with monadic computations extended with an aliasing, a parallel composition
  and a choice. We specify laws about these operations to guarantee that @docase@ keeps the familiar 
  semantics of pattern matching using @case@ (Section \ref{sec:reasoning}).

\item To demonstrate the usefulness of the extension, we consider parsing
  (Section \ref{sec:intro-combining-parsers}), GUI programming using events
  (Section \ref{sec:intro-choosing-events}), lightweight concurrency
  (Section \ref{sec:intro-commit-poorman}), and a parallelism monad with
  support for speculative parallelism (Section \ref{sec:intro-aliasing-parallel}).

\item The type of the above computations is captured by a \ident{Joinad} type class (Section 
  \ref{sec:extension-typeclass}). It relates type classes that have been already proposed for 
  Haskell. Based on our experience, we propose and discuss several adjustments to the Haskell base 
  library and laws required by the type classes we combine (Section \ref{sec:proposals}). 

\item A joinad is an abstract computation that extends monads with three operations.
  Deriving the laws about the three operations (Section \ref{sec:laws}) reveals that the two
  operations form an algebraic structure known as \textit{near-semiring}.

\end{itemize}
The following section shows an example that demonstrates the usefulness of @docase@ in the context 
of parallel programming.

% ==================================================================================================

\section{Motivating example}
\label{sec:motivation}

Consider the following problem: we are given a tree with values in leaves and we want to test
whether a predicate holds for all values in the tree. This can be implemented as a recursive function:

\begin{code}
forall :: (a -> Bool) -> Tree a -> Bool

forall p (Leaf v)           = p v
forall p (Node left right)  = forall p left && forall p right
\end{code}
The execution of the two recursive calls could proceed in parallel. Moreover, when one of the 
branches completes returning @False@, it is not necessary to wait for the completion of the other 
branch as the overall result must be @False@.

Running two branches in parallel can be specified easily using strategies \cite{strategies-new, strategies-old}, 
but adding shortcircuiting behaviour is challenging. Using the @docase@ notation and a monad for 
parallel programming, the problem can be solved as follows:

\begin{code}
forall :: (a -> Bool) -> Tree a -> Par Bool

forall p (Leaf v)           = return (p v)
forall p (Node left right)  = 
  docase_ (forall p left, forall p right) of
    (False, ?)    -> return False
    (?, False)    -> return False
    (allL, allR)  -> return (allL && allR)
\end{code}
The function builds a computation that explicitly specifies the evaluation order using the
parallelism monad |Par| \cite{parmonad} extended with the support for non-deterministic choice 
operator \cite{parmonad-cancellation}.

To process sub-trees in parallel, the snippet constructs two computations (of type |Par Bool|) 
and uses them as arguments of @docase@. Patterns in the alternatives correspond to individual
computations. A special pattern @?@ denotes that a value of the monadic computation does not 
have to be available for the alternative to be selected. When the processing of the left subtree 
completes and returns @False@, the first alternative can be selected immediately, because the 
result of the second computation is not required.

If the result of the left subtree is @True@ and the right one has not completed, none of 
the alternatives is enabled. After the right subtree is processed, one of the last two alternatives 
can be selected. The choice operator added to the |Par| monad is non-deterministic, so the 
programmer needs to provide alternative clauses that produce the same result in case of race.
We return to this topic in Section~\ref{sec:intro-aliasing-parallel}, but it is, indeed, 
the case in the above example.

The selection between alternative clauses is done using the \textit{choice} operator. The result 
of each computation is used in two independent alternatives. Evaluating the argument repeatedly 
would defeat the purpose of @docase@, so the translation uses \textit{aliasing} operator to avoid 
this. The last alternative combines two computations, which is achieved using \textit{parallel
composition}.

The translation of the @docase@ notation is more complex than of the @do@ notation. This is not
a bad thing---the notation can be used to write programs that would otherwise be not
only less syntactically elegant, but also more complex. In the above example, developers would
more likely write the solution in a more imperative style shown in Appending~\ref{sec:appendix-shortcircuit}.
The length of the explicit version is 21 lines compared to 6 line in the version above. 

% ==================================================================================================

\section{Introducing docase}
This section introduces @docase@ notation using four examples. We first consider @docase@
expressions with just a single alternative and then gradually add remaining features. A formal 
definition is presented in Section~\ref{sec:extension}.

% --------------------------------------------------------------------------------------------------

\subsection{Parallel composition of parsers}
\label{sec:intro-combining-parsers}
Parsers are a common example of monads. A parser is a function. When supplied with an input, it 
returns a parsed value and the unconsumed rest of the input. The following definition mostly follows
the one by Hutton and Meijer \cite{parsingtutorial}:

\begin{code}
newtype Parser a = P (String -> [(a, Int, String)])
\end{code}
Compared to standard parsers, there is one notable difference. In addition to the parsed result
and unconsumed input, the result also contains @Int@ value, which denotes the number of consumed
characters. This will be needed later.
A parser can be made an instance of \ident{Monad} to allow sequential composition and an instance 
of \ident{MonadPlus} to support choice. The definitions for the above @Parser@ type are standard. 
A more interesting question is, what does a parallel composition of parsers mean:

\begin{code}
mzip :: Parser a -> Parser b -> Parser (a, b)
\end{code}
The @mzip@ operation has been independently proposed as a way to generalize parallel list 
comprehensions to work with monads \cite{bringbackmc}. We return to this topic later in Section 
\ref{sec:proposals-monadzip}. For some monads, the operation can be implemented in terms of |>>=| 
and @return@. An additional requirement about the implementation is that it should be symmetric, 
so this is possible only for commutative monads such as @Reader@. For parsers, the @mzip@ operation 
can parse the input using both parsers and then and return all combination of values such that the 
two parsers consumed the same number of input characters. The meaning of this operation is that it 
creates a parser for a language that is an \textit{intersection} of languages described by the 
two parsers.

This implementation of @mzip@ for parsers is shown in Figure \ref{fig:parser-monadzip}. It
applies the two parsing functions to the same input and then returns all combinations for which
the predicate |num1 == num2| holds. Instead of comparing the number of consumed characters, the 
implementation could compare the tails, but that would be inefficient and it would not work for
infinite inputs. The intersection of parsers is very useful for encoding validation of 
inputs.

\begin{figure}
\label{fig:parser-monadzip}
\begin{code}
instance MonadZip Parser where 
  mzip (P p1) (P p2) = P $ \input -> --$
    [ ((a, b), num1, tail1) |
         (a, num1, tail1) <- p1 input,
         (b, num2, tail2) <- p2 input, num1 == num2 ]
\end{code}
\caption{Instance of \ident{MonadZip} for parsers}
\end{figure}

\paragraph{Example.}
Cambridge phone numbers consist of 10 symbols, contain only digits and start with 
a prefix 1223. Using standard parser combinators and the new syntax, we can write the 
following parser:

\begin{code}
validPhone = docase_
   sp_ many (sat isDigit), 
       replicateM 10 item,
       startsWith (string "1223") of
  num, _, _ -> return num
\end{code}
In this snippet, the @docase@ construct is given three parsers as arguments. Patterns specify
that all three parsers should succeed and produce a value. The @_@ pattern is different to the
@?@ pattern from the previous example. The @?@ pattern is special and it specifies that a value
is not required. On the other hand, @_@ is a standard Haskell pattern that requires
a value, but ignores it afterwards. The reason why the snippet ignores the values produced by the
last two parsers is that all parsers consume the same input and return the same value.
The body of the alternative creates a parser that succeeds and returns the phone number as
a string.

It is possible to construct a single parser that recognizes valid Cambridge phone number without using
the intersection. The point of the example is that the intersection (and @docase@) provide a
very elegant way for combining several independent rules, which makes the validation code
elegant and easy to understand.

\paragraph{Desugaring.}
The desugaring of @docase@ translates the previous snippet to the code below. In the 
translation, the @mzip@ operation is written as |mzip_|. The reason for this will become clear
when we discuss the algebraic theory behind joinads (Section \ref{sec:theory-algebra}).

\begin{code}
validPhone = 
    ((many (sat isDigit) mzip_ times item 10) 
        mzip_ startsWith (string "1223")) >>= \x ->
            case x of ((num, _), _) -> return num
\end{code}
The actual translation includes several additional features that will be explained later.
The expression shown above combines all arguments of @docase@ using the |mzip_| operation. The 
result is a combined value of type |Parser ((String, String), String)|. The value is passed
as an input to |>>=|. The continuation matches the tuple value against a tuple pattern composed 
from the individual patterns of the @docase@ alternative. In this example, the pattern never fails, 
so other cases are omitted. The body of the continuation creates a parser that succeeds and 
returns the parsed valid phone number.

Next, consider a case where @docase@ has multiple alternatives, but each contains only a single
binding (a pattern other than @?@).

% --------------------------------------------------------------------------------------------------

\subsection{Choosing between events}
\label{sec:intro-choosing-events}

The examples in this section are based on the imperative stream monad developed by Scholz
\cite{imperative-streams}. Imperative streams are ``a generalization of the IO monad suitable for
synchronous concurrent programming''. A computation |Imp a| produces zero or more
values of type |a| and performs side-effects at certain (discrete) times. Our example uses 
a simplified model of \textit{event streams} with type |Evt a| that do not allow side-effects.

Event streams can be viewed as functions that take a time when they are started and return a list of 
time value pairs representing the occurrences of the event. They are instances of the \ident{Monad} 
type class. The @return@ operation creates an event stream that occurs exactly once at the time when 
it was started. The behaviour of the |>>=| operation is as follows: When the input event occurs, the 
event stream returned by |>>=| starts producing the occurrences of the event stream generated by the 
continuation until the next occurrence of the input event.

In addition to the operations of \ident{Monad}, event streams can also implement a monadic or--else
operation representing a choice:

\begin{code}
morelse :: Evt a -> Evt a -> Evt a
\end{code}
The resulting event stream occurs whenever any of the two arguments occur. When both of the 
arguments occur at the same time, then the returned value is the value produced by the first (left)
argument. As explained later (Section \ref{sec:reasoning}), this left bias of the operation is 
required by a law about @morelse@.

\paragraph{Example.} Assume that the user can create objects by clicking on some window. A click 
creates a different object when the \ident{Shift} key is pressed. The user interface provides event 
streams @shiftDown@ and @shiftUp@ that occur when \ident{Shift} is pressed and released; an event 
stream @load@ occurs once when the application starts and @mouseClick@ occurs when the button is pressed.

The following snippet creates an event stream of type |Evt Bool| that occurs each time a mouse 
button is clicked. The value carried by the event is a flag denoting whether \ident{Shift} was
pressed:

\begin{code}
shiftClicks = docase_ load, shiftUp, shiftDown of
  a, ?, ?  -> fmap (const False) mouseClick
  ?, u, ?  -> fmap (const False) mouseClick
  ?, ?, d  -> fmap (const True) mouseClick
\end{code}
The @docase@ construct takes three event streams. When one of them produces a value,
the construct selects one of the alternatives and @shiftClicks@ starts producing values generated
by the alternative (@True@ or @False@ whenever mouse is clicked). Each of the alternatives matches
on just a single event stream (using a variable pattern) and ignores the values of other event 
streams using @?@. The variables bound by the patterns are not used, so we could use @_@ pattern, 
but naming the variables makes the example easier to follow.

\paragraph{Desugaring.}
The example is translated to the code below. Because each alternative binds only
on a single argument, the translation doesn't need the @mzip@ operation. Similarly, to the 
previous desugared example, the @morelse@ operation is abbreviated as |morelse_|.

\begin{code}
shiftClicks = 
  (load >>= \a -> fmap (const False) mouseClick)  morelse_
  (shiftUp >>= \u -> fmap (const False) mouseClick) morelse_
  (shiftDown >>= \d -> fmap (const True) mouseClick)
\end{code}
The translator processes alternatives independently and then merges them using |morelse_|. The 
event stream that the alternative binds on is passed as the first argument to |>>=|. The function 
used as the second argument contains the body of the alternative. 
The example is simplified, because patterns in the alternatives do not fail. If the pattern matching
failed, the event stream should continue behaving according to the last selected alternative. To
encode this behaviour, the translation needs one more extension (Section \ref{sec:intro-commit-poorman}).

\begin{figure}
\label{fig:imperativestream-semantics}
\begin{code}
^sm(unit v)^sub(s)       = \t -> if s == t then Just v else Nothing

^sm(a mzip_ b)^sub(s)     = \t -> case (^sm(a)^sub(s) t, ^sm(b)^sub(s) t) of
  sp_ sp_  (Just v^sub(1), Just v^sub(2)) -> Just(v^sub(1), v^sub(2)); sp_ _ -> Nothing

^sm(a morelse_ b)^sub(s)  = \t -> case (^sm(a)^sub(s) t, ^sm(b)^sub(s) t) of
           (Just v^sub(1), _) -> Just v^sub(1); sp_ (_, o^sub(2)) -> o^sub(2)

^sm(a >>= f)^sub(s)      = \t -> case (last t smp_ ^sm(a)^sub(s)) of
           (Just(t^sub(1), v^sub(1))) -> ^sm(f v^sub(1))^sub(t^sub(1)) t; sp_ _ -> Nothing
  where  last 0 _   = Nothing
         last t sf  = case sf t of  Just v -> Just (t, v)
                                    _ -> last (t - 1) sf
\end{code}
\caption{Semantics of imperative streams}
\end{figure}

\paragraph{Semantics.} Showing a complete implementation of event streams is beyond the scope of
this article. We present a semantics that defines the implementation and can be used to verify 
that the operations obey joinad laws. The semantics follows the original definition of imperative 
streams \cite{imperative-streams}. Instead of using lists, we model event occurrences using a 
partial function:

\begin{code}
^sm(-) :: Evt a -> T -> T -> Maybe a
\end{code}
The time $T$ is a discrete value. When applied to a starting time $t \in T$, the semantic function 
gives a partial function that returns |Just v| if the event occurs at the specified time.
The semantics of \ident{Monad} operations, |morelse_| and also |mzip_| is given in 
Figure \ref{fig:imperativestream-semantics}.

The event stream created by @unit@ produces a value at the time when it is started.
The semantics of |mzip_| and |morelse_| follow a similar pattern. At given time, they combine 
both, or take the leftmost value if the required values are available. Finally, |>>=| behaves as 
an event stream generated by the last occurrence of the input event.

Using this semantic model, we could derive an implementation using the techniques developed recently 
for functional reactive programming (FRP) by Elliott \cite{push-pull-frp}. Compared to other 
approaches, imperative streams give a simple model based just on discrete events. The other aspect 
of FRP---working with continuous values---could be added similarly, but it would not notably benefit 
from the @docase@ notation, because its \textit{parallel composition} and \textit{choice} operations
are trivial.

% --------------------------------------------------------------------------------------------------

\subsection{Aliasing parallel computations}
\label{sec:intro-aliasing-parallel}

In the previous two examples, each argument of @docase@ was bound only in one alternative.
This is not a requirement. Indeed, the @forall@ function from the introduction started two 
computations and matched their values against two alternatives (one that implemented shortcircuiting
and the other handled the general case). For monads representing computations, this requires a special handling.

The problem can be demonstrated using a parallelism monad. A value of type
|Par a| represents a computation that can be evaluated (using some parallel evaluator) to get a 
value of type |a|. Parallel computations are instances of \ident{Monad}. The @return@
operation creates a computation that immediately evaluates to a specified value and |>>=| 
evaluates the argument and then evaluates the result produced by a continuation.
The |Par a| type also supports operations from the previous two sections. The @mzip@ operation 
starts two computations in parallel and produces a value when they both complete; @morelse@
represents a non-deterministic choice and completes when the first of the two computations 
produce a value.

\begin{figure}
\begin{code}
^sm(unit v)    = \t -> (t, v)

^sm(mzero)      = \t -> (inf_, bot_)

^sm(a mzip_ b)  = \t -> (max t^sub(1) t^sub(2), (v^sub(1), v^sub(2)))
	where  (t^sub(1), v^sub(1)) = ^sm(a)^sub(t) and_ (t^sub(2), v^sub(2)) = ^sm(b)^sub(t)

^sm(a morelse_ b)  = \t -> (min t^sub(1) t^sub(2), v)
	where  (t^sub(1), v^sub(1)), (t^sub(2), v^sub(2)) = ^sm(a)^sub(t), ^sm(b)^sub(t)
	       v | t^sub(1) <= t^sub(2) = v^sub(1) | otherwise = v^sub(2)

^sm(a >>= f)  = \t -> ^sm(b)^sub(s)
	where  (s, v) = ^sm(a)^sub(t) and_ b = f v

^sm(mduplicate a)  = \t -> (\t^sub(2) -> (max t^sub(1) t^sub(2), v))
	where (t^sub(1), v) = ^sm(a)^sub(t)
\end{code}
\label{fig:future-semantics}
\caption{Semantics of futures}
\end{figure}

\paragraph{Example.} Assume that we have some calculation that uses a main function and two alternative
heuristic functions. In order to continue, we need the result of the main function and one heuristic.
Using @docase@, this can be written as follows:

\begin{code}
calcAlt inp = docase_ calc inp, alt1 inp, alt2 inp of
  a, ?, b -> return (a, b)
  a, b, ? -> return (a, b)
\end{code}
Note that the first argument is bound to a variable @a@ in both of the alternatives. The desired 
operational meaning is that the expression starts all three computations in parallel and then waits 
until computations required by some alternative complete. Using the logic described so far, the 
snippet could be translated as follows:

\begin{code}
calcAlt inp = 
  (calc inp) mzip_ (alt1 inp) >>= \ (a, b) -> return (a, b) morelse_
  (calc inp) mzip_ (alt2 inp) >>= \ (a, b) -> return (a, b)
\end{code}
This doesn't give the required behaviour. The code creates a computation that 
starts four tasks -- the two alternative heuristics and \textit{two times} the main
computation. Eliminating a common subexpression |calc inp| does not solve the problem. The value 
|Par a| obtained from |calc inp| represents a recipe for creating a computation, as opposed to 
a task with a mutable variable that is eventually set. When used repeatedly, it repeatedly 
starts a new computation.

\paragraph{Desugaring.} To get the desired semantics, we need another operation. The operation
represents aliasing of monadic computations. It can be best explained by looking at the type 
signature together with the implementation for the |Par a| monad:

\begin{code}
mduplicate :: Par a -> Par (Par a)
mduplicate m = do 
  v <- spawn m
  return (get v)
\end{code}
The operation builds and returns (inside the monad) a new monadic value that represents 
an aliased reference to the computation. The implementation starts a given computation 
using the @spawn@ function. The function returns a mutable variable that will contain the result 
of the computation when it completes. The |Par a| value is created using the @get@ function,
which builds a computation that blocks until the variable is set.

Equipped with this operation, the desugaring can create an aliased monadic computation for each
of the @docase@ arguments and then combine the aliased computations:

\begin{code}
calcAlt inp = 
  mduplicate (calc inp) >>= \c0 ->
  mduplicate (alt1 inp) >>= \c1 ->
  mduplicate (alt2 inp) >>= \c2 ->
  c0 mzip_ c1 >>= \ (a, b) -> return (a, b) morelse_
  c0 mzip_ c2 >>= \ (a, b) -> return (a, b)
\end{code}
This version gives the desired operational behaviour. The three computations are started in 
the implementation of @mduplicate@. The body then composes computations that represent
just aliases (by waiting on a variable).

\paragraph{Semantics.} To describe the example more formally, we present a simple semantics
of the parallelism monad. It can be used to verify that the laws required about joinad operations 
(Section \ref{sec:laws}) hold for the |Par a| type. A computation is modelled as a function that
takes a time when the computation is started and returns a time when it completes together with
the result:

\begin{code}
^sm(-) :: Par a -> T -> (T, a)
\end{code}
A semantics expressed in terms of this function is shown in Figure~\ref{fig:future-semantics}.
Operations of \ident{Monad} as well as |mzip_| and |morelse_| behave as already informally described.
The @mduplicate@ operation applies the semantic function to a given computation with the starting 
time of @mduplicate@ as an argument. The produced computation finishes then completes either at 
the completion time or at the time when it is created, whichever happens later.

The semantics doesn't capture the fact that evaluating computation multiple times requires 
additional CPU resources. Therefore, it doesn't demonstrate the importance of @mduplicate@ for
the parallelism monad, but it is useful for considering joinad laws. The next section
informally describes a variation that enables effects in concurrent computations, which makes
the importance of @mduplicate@ more obvious. It prevents the duplication of effects. 

\paragraph{Nondeterminism.} The semantics presented above is deterministic, but in reality, this 
is not the case. We could add small $\delta$ to all operations involving time. The interesting 
case is the |morelse_| operation, where the value (second element of the tuple) depends on the time.
This means that the operation introduces non-determinism.

To keep programs simple and deterministic, we follow the approach used by Elliott \cite{push-pull-frp} 
for the unambiguous choice (@unamb@) operator. The @morelse@ operation can be used only when the 
two arguments are \textit{compatible}:

\begin{code}
compatible a b = forall_ t {-". \ "-} (t^sub(1) == inf_) || (t^sub(2) == inf_) || (v^sub(1) == v^sub(2))
	where  (t^sub(1), v^sub(1)) = ^sm(a)^sub(t) and_ (t^sub(2), v^sub(2)) = ^sm(b)^sub(t)
\end{code}
When started at the same arbitrary time, the two operations have to produce the same value when they
both complete. An operation that never completes can be created using the @mzero@ operation, which 
is explained in the next section. The condition could be reformulated for alternatives of the @docase@
expression formally, but it is not difficult to see that the condition holds for the motivating 
example from Section \ref{sec:motivation}.

% --------------------------------------------------------------------------------------------------

\subsection{Committing to a concurrent alternative}
\label{sec:intro-commit-poorman}

The last simplification that we made in the previous example was that the patterns of @docase@
alternatives never failed. We demonstrate this aspect using a monad based on Poor Man's Concurrency 
Monad developed by Claessen \cite{poorman}. A value of type |Concur m a| represents a computation 
that will eventually produce a value of type |a| and may produce effects in a monad |m| along the way. 

The monad is similar to parallel computations from the previous section. For example, assume
that the underlying monad is a writer monad using monoid |M| to keep the state. The semantics from 
the previous section could be extended by adding state to the result:

\begin{code}
^sm(-) :: Concur m a -> T -> (T, a, M)
\end{code}
The semantics from Figure \ref{fig:future-semantics} can be extended in a straightforward way to use
this function. Unlike |Par a|, the implementation of |Concur m a| does not actually run computations in 
parallel. It emulates concurrency by interleaving of steps. This means that the |Concur m a| monad is 
deterministic and the time |T| represents the number of primitive steps. In order to support patterns 
that may fail, the type also provides an operation @mzero@ representing a failure:

\begin{code}
mzero :: Concur m a
\end{code}
The operation is defined for standard monads that implement the \ident{MonadPlus} type class. For the
Poor Man's Concurrency monad, the operation creates a computation that never produces a value.

\paragraph{Example} Consider a function that concurrently downloads data from two servers. When some
data is not available, the function returns @Nothing@. When both servers produce a value, the function
returns @Just@ containing a tuple, but only when the values are compatible. When data is incompatible,
the function should fail.

\begin{code}
downloadBoth :: Concur IO Bool
downloadBoth = docase_ dl1, dl2 of
  Just a, Just b -> do
    lift (print ``Got both values'')
    if compatible a b then return True else mzero
  _, _ -> do
    lift (print ``Some value missing'')
    return False
\end{code}
The first alternative of @docase@ binds on both computations and matches when both produce @Just@ value.
The body prints a message (the @lift@ function lifts a value |IO a| into |Concur IO a|) and then 
returns @True@ or fails by returning @mzero@. The second alternative handles any values and returns 
@False@ after printing a message.

A simple way to desugar failing patterns (suggested by Tullsen \cite{firstlcasspats}) is to return 
@mzero@ when pattern matching fails. When it succeeds, the original body is returned. For simplicity,
we omit the aliasing using @mduplicate@:

\begin{code}
(dl1 mzip_ dl2 >>= \t -> case t of 
    (Just a, Just b) -> do smp_ ...
    _ -> mzero) morelse_
(dl1 mzip_ dl2 >>= \t -> do smp_ ...)
\end{code}
An intuitive expectation about pattern matching that we want to keep for joinads is that 
|->| behaves as a \textit{commit point}. Once arguments match patterns of some alternative, the 
code will execute this alternative and not any other. However, this is not the case with the
desugaring above. When the function downloads two incompatible values, it prints ``Got both values''.
Then it fails and starts executing the second clause, printing ``Some values missing''.

\paragraph{Desugaring} To get the desired behaviour, the desugaring needs to add an additional
level of wrapping. Instead of just returning the body, we wrap the body using @return@:

\begin{code}
(dl1 mzip_ dl2 >>= \t -> case t of 
    (Just a, Just b) -> return (do smp_ ...)
    _ -> mzero) morelse_
(dl1 mzip_ dl2 >>= \t -> 
    return(do smp_ ...)) >>= id
\end{code}
The cases where the pattern matching succeeded now contain just a call to @return@ with the body
of the alternative as an argument and the cases where the pattern matching fails contain @mzero@.
This means that the type of values aggregated using |morelse_| is |m (m a)|. Additionally,
all of them are either of the form |m >>= return . f| or |m >>= mzero|.
The result of |morelse_| has the same type as its arguments, so the overall result also has a type
|m (m a)|. It represents a monadic value that wraps (or produces) body (or bodies) that have
been selected. To create a computation that actually runs the body, the desugaring inserts 
|>>= id| at the end of the translated expression. 

% ==================================================================================================

\section{Language extension}
\label{sec:extension}

This section presents the @docase@ extension formally. It introduces the syntax and type classes
that a type must implement to support the syntax. Then we look at typing rules and present a 
translation in terms of operations provided by the \ident{Joinad} type class.

% --------------------------------------------------------------------------------------------------

\subsection{Syntactic extension}
\label{sec:extension-syntax}

The extension adds an additional case to Haskell expression $e$. It also 
defines two additional syntactic categories to extend the language of patterns (symbol $w$) that 
can be used in alternatives (symbol $a$) of the @docase@ construct:

\begin{code}
p  =  x | (p^sub(1), ..., p^sub(n)) | ...          {-"\text{Ordinary patterns}"-}

w  =  ?                                            {-"\text{Monadic ignore pattern}"-}
   |  p                                            {-"\text{Monadic binding pattern}"-}

a  =  w^sub(1), ..., w^sub(n) -> e                 {-"\text{Docase alternative (clause)}"-}

e  =  docase_ e^sub(1), ..., e^sub(n) of  sp_ sp_  {-"\text{Docase expression with}"-}
        a^sub(1); ...; a^sub(k)                    {-"\text{$k \geq 1$ alternatives}"-}
\end{code}
The alternatives of @docase@ are similar to the alternatives of @case@. In addition to ordinary 
Haskell patterns, alternatives can match monadic computations against the special 
\textit{ignore pattern} written as @?@. The syntax allows this by duplicating the syntactic
category of alternatives and defining a new category for @docase@ patterns.

% --------------------------------------------------------------------------------------------------

\subsection{Joinad type class}
\label{sec:extension-typeclass}

The @docase@ syntax operates on values of some type |m a| that is an instance of a \ident{Joinad} 
type class. The type class provides operations required by the @docase@ translation. The Figure 
\ref{fig:joinad-classes} shows the definition of \ident{Joinad}. The definition just combines several 
classes that already exist in various Haskell extensions and packages. 

\begin{itemize}
\item \ident{MonadZero} and \ident{MonadOr} are defined in a \ident{MonadPlus} reform proposal
\cite{monadplusreform}. The aim of the proposal is to distinguish between cases when the (monoidal)
operation is unbiased (\ident{MonadPlus}) and when it has a left bias (\ident{MonadOr}). For joinads,
we require left bias, but we express the law slightly differently (Section \ref{sec:laws-monador}).

\item \ident{MonadZip} is defined by a GHC extension that adds \textit{monad comprehensions}
\cite{bringbackmc, comprefun}. The extension adds new expressive power that is not available with 
the @do@ notation \cite{parcomprefun}. The @docase@ syntax uses the type class in a similar way as
\textit{parallel monad comprehension} and provides similar expressivity using a syntax similar to 
the @do@ notation.

\item \ident{MonadDuplicate} is similar to the \ident{Extend} type class from the comonad 
package \cite{comonadpkg}. The only difference is that we require the type to also be a monad,
which allows us to define a default implementation of the @mduplicate@ operation. 

\end{itemize}
The laws that are required for the operations are discussed in more details later in Section
\ref{sec:reasoning} and the category theoretical background of the definitions is discussed in
Section \ref{sec:theory}. Before looking at the theory, the next two section complete the 
specification of the language extension.

\begin{figure}
\begin{code}
class Monad m => MonadZero m where
   mzero :: m a

class MonadZero m => MonadOr m where
   morelse :: m a -> m a -> m a

class Monad => MonadZip m where
  mzip :: m a -> m b -> m (a, b)

class Monad m => MonadDuplicate m where
  mduplicate :: m a -> m (m a)
  mduplicate m = return m

class (MonadDuplicate m, MonadZip m, MonadOr m) 
  => Joinad m
\end{code}
\caption{ The definition of \ident{Joinad} type class. }
\label{fig:joinad-classes}
\end{figure}

% --------------------------------------------------------------------------------------------------

\subsection{Typing rules}
\label{sec:extension-typing}

Similarly to other syntactic sugar in Haskell \cite{groupordercompre}, the @docase@ expression 
is type-checked before the translation. The typing rules are shown in Figure \ref{fig:typing-rules} 
and are defined in terms of three judgements.

The judgement $\vdash w : \tau \Rightarrow \Delta$ for patterns is similar to the one used by
Wadler and Peyton Jones \cite{groupordercompre}. It specifies that a pattern $w$ of type $\tau$ binds 
variables of the environment $\Delta$. An ignore pattern does not bind any variables (\textsc{Ign}); 
a variable pattern binds a single variable (\textsc{Var}) and a tuple pattern binds the union of 
variables bound by sub-patterns (\textsc{Tup}).

The judgement $\Gamma, \alpha, \bar{\tau} \vdash a \leadsto \sigma $ is more interesting. It 
checks the type of an individual alternative of the @docase@ construct. The judgement is provided 
with an environment $\Delta$, a type $\alpha$ which is a \ident{Joinad} and a list of types of 
@docase@ arguments. It type-checks the alternative and yields a type of values produced by the
body of the alternative (\textsc{Alt}). Two aspects of the rule do not directly follow 
from the syntax. Firstly, the body $e$ of the alternative must have the same monadic type 
$\alpha$ (of kind |* -> *|) as the @docase@ arguments. Secondly, at least one of the patterns $w$
of every alternative $a$ must be a \textit{binding pattern}. The reasons for these two constraints 
will become obvious when we discuss the translation.

Finally, $\Gamma \vdash a : \tau$ extends the standard type-checking procedure
for Haskell expressions with a rule for @docase@ (\textsc{Jon}). When the type of arguments is 
a \ident{Joinad} type $\alpha$ (of kind |* -> *|) applied to some type argument and all 
alternatives yield the same return type $\sigma$, then the overall type of the expression is 
$\alpha \ \sigma$.


\begin{figure}
\begin{equation*}
\fbox{$\vdash w : \tau \Rightarrow \Delta$}
\end{equation*}

\begin{equation*}
\inferrule
  { }
  {\vdash \text{?} : \tau \Rightarrow \{ \}}
\quad(\textsc{Ign}) \ \ \ \ \ \ \ \
\inferrule
  { }
  {\vdash x : \tau \Rightarrow \{ x : \tau\}}
\quad(\textsc{Var})
\end{equation*}

\begin{equation*}
\inferrule
  {\vdash w_i : \tau_i \Rightarrow \Delta_i}
  {\vdash (w_1, \ldots, w_n) : (\tau_1, \ldots, \tau_n) \Rightarrow \Delta_1 \cup \ldots \cup \Delta_n}
\quad(\textsc{Tup})
\end{equation*}

\begin{equation*}
\fbox{$\Gamma, \alpha, \bar{\tau} \vdash a \leadsto \sigma $}
\end{equation*}

\begin{equation*}
\inferrule
  {\vdash w_i : \tau_i \Rightarrow \Delta_i \\ \exists i : w_i \neq \text{?} 
     \\\\ \Delta_1 \cup \ldots \cup \Delta_n \vdash e : \alpha \ \sigma  }
  {\Gamma, \alpha, \bar{\tau} \vdash (w_1, \ldots, w_n) \rightarrow e \leadsto \sigma }
\quad(\textsc{Alt})
\end{equation*}

\begin{equation*}
\fbox{$\Gamma \vdash a : \tau$}
\end{equation*}

\begin{equation*}
\inferrule
  {\langle\textbf{Joinad} \ \alpha \rangle \\ \Gamma \vdash e_i :\alpha \ \tau_i 
    \\ \Gamma, \alpha, \bar{\tau} \vdash a_i \leadsto \sigma }
  {\Gamma \vdash \textbf{docase} \ \bar{e} \ \textbf{of} \ a_1; \ldots; a_n : \alpha \ \sigma}
\quad(\textsc{Jon})
\end{equation*}
\caption{ Semantics-preserving transformations for docase. }
\label{fig:typing-rules}
\end{figure}


% --------------------------------------------------------------------------------------------------

\subsection{Translation}
\label{sec:extension-translation}

After type-checking, the @docase@ notation is translated to applications of functions provided by the 
\ident{Joinad} type class. As demonstrated by the previous examples, the desugaring is slightly more 
complicated than the @do@ notation. We define it using two functions:

\begin{code}
d^tsm(-)  :: e -> e
c^tsm(-)  :: a -> [id] -> e
\end{code}
The first function takes an expression. If the argument is the @docase@ expression, the function
produces an expression that does not contain @docase@ at the top-level. The second function is used
for translating alternatives of @docase@. It takes a list of identifiers that refer to the arguments
of the @docase@ expression. The translation is defined by the following two rules:

\begin{code}
d^tsm(docase_ e^sub(1), ..., e^sub(n) of a^sub(1); ...; a^sub(k)) =
  mduplicate e^sub(1) >>= \v^sub(1) -> sp_ ... sp_ mduplicate e^sub(n) >>= \v^sub(n) ->
  (c^tsm(a^sub(1)) [v^sub(1), ..., v^sub(n)] morelse_ ... morelse_ c^tsm(a^sub(n)) [v^sub(1), ..., v^sub(n)]) >>= id

lsp_

c^tsm(w^sub(1), ..., w^sub(n) -> e) [v^sub(1), ..., v^sub(n)] = 
  v^sub(1) mzip_ ... mzip_ v^sub(m) >>= \x -> case x of 
    (p^sub(1), ..., p^sub(m))  -> return e
    otherwise                  -> mzero

  where  [ (p^sub(1), v^sub(1)), ..., (p^sub(m), v^sub(m)) ] = 
           [ (w^sub(i), v^sub(i)) | i <- 1 ... n, w^sub(i) neq_ ?]
\end{code}
The expressions given as arguments to @docase@ are first passed to the @mduplicate@ function, which
constructs a value of type |m (m a)|. When passed to |>>=|, the continuation is called with a value
|v^sub(i)| that represent an aliased computations. The variables are passed to |c^tsm(-)| when 
translating alternatives. Alternatives are translated to values of type |m (m a)| that represent 
monadic values carrying bodies to be executed. The monadic values are then combined using the |morelse_| 
operations. The result of this combination is still a value of type |m (m a)|, so the last binding 
passes it to the identity function to execute the body of the selected alternative.

To translate an alternative, we identify which of the arguments are matched against
a binding pattern. These computations are combined using the |mzip_| operation. The resulting 
computation carries tuples such as |(a, (b, c))|. As discussed later, the |mzip_| operation is 
associative, so the order does not matter. Values carried by the combined monadic computation 
are matched against a pattern re-constructed from \textit{binding patterns} of the alternative. 
When a value matches, the body is wrapped and returned using @return@. If the value does not match, 
the alternative reports a failure using @mzero@.

% ==================================================================================================

\section{Reasoning about joinads}
\label{sec:reasoning}

The @docase@ syntax intentionally resembles the @case@ construct. To turn this syntactic similarity 
into a useful intuition for developers, we would also like to guarantee that the semantics of 
@docase@ resembles the semantics of @case@. 

This intuition can be formalized by specifying that code transformations in Figure~\ref{fig:transformations} 
preserve the semantics. The syntactic transformations motivate laws about primitive joinad operations 
that are discussed in the next section. When the primitive operations obey certain laws, the syntactic
transformations are correct. We proved that this is the case using the Coq theorem prover. 
\footnote{The script is available at http://www.cl.cam.ac.uk/\textasciitilde tp322/TBD}

\begin{figure}
\begin{code}
lsp_
{-"\text{(1) Binding equivalence}"-}
  docase_ m of v -> e sp_ == sp_ do v <- m; e

lsp_
{-"\text{(2) Argument ordering}"-}
  docase_ m^sub(1), ..., m^sub(n) of
    w^sub(1,pi_ 1) ... w^sub(1,pi_ n) -> e^sub(1); ...
    w^sub(k,pi_ 1) ... w^sub(k,pi_ n) -> e^sub(k)
  {-"\text{(are equivalent for any permutation $\pi$ of $n$ numbers)}"-}

lsp_
{-"\text{(3) Clause ordering}"-}
    docase_ m of v -> e^sub(1); sp_ v -> e^sub(1)
==  docase_ m of v -> e^sub(1)

lsp_
{-"\text{(4) Alternative noninterference}"-}
    docase_ m of v       -> e^sub(1); sp_ pfail_  -> e^sub(2)
==  docase_ m of pfail_  -> e^sub(2); sp_ v       -> e^sub(1)
==  docase_ m of v       -> e^sub(1)

lsp_
{-"\text{(5) Argument noninterference}"-}
    docase_ m, mzero of v, ?                -> e^sub(1);      v^sub(1), v^sub(2)  -> e^sub(2)
==  docase_ mzero, m of v^sub(1), v^sub(2)  -> e^sub(2); sp_  ?, v                -> e^sub(1)
==  docase_ m of v -> e^sub(1)

lsp_
{-"\text{(6) Matching units}"-}
    docase_  return e^sub(1), return e^sub(2) of v^sub(1), v^sub(2) -> e
==  case (e^sub(1), e^sub(2)) of (v^sub(1), v^sub(2)) -> e

lsp_
{-"\text{(7) Matching images}"-}
    docase_ map f e^sub(1), map g e^sub(2) of v^sub(1), v^sub(2) -> e
==  docase_ e^sub(1), e^sub(2) of u^sub(1), u^sub(2) -> e[v^sub(1) <- f u^sub(1), v^sub(2) <- g u^sub(2)]

lsp_
{-"\text{(8) Matching duplicate}"-}
    docase_ a, a of u, v -> e
==  docase_ a of u -> e[v <- u]

lsp_
{-"\text{(9) Flattening}"-}
    docase_ m, n^sub(1), n^sub(2) of 
      v, v^sub(1), ? -> e^sub(1); sp_ v, ?, v^sub(2) -> e^sub(2)
==  docase_ m, (docase_ n^sub(1), n^sub(2) of
      v^sub(1), ? -> return sp_ \v -> e^sub(1); 
      ? , v^sub(2) -> return sp_ \v -> e^sub(2)) of v, f -> f a
\end{code}

\caption{ Semantics-preserving transformations for docase. }
\label{fig:transformations}
\end{figure}

\begin{itemize}
\item \textit{Binding equivalence} describes a degenerate case in which pattern matching uses
  a single alternative that always succeeds. Additional complexities of the @docase@ translation
  should not affect the meaning of monadic binding.
  
\item \textit{Argument ordering} specifies that the order in which arguments and patterns are
  specified does not affect the meaning. This equation implies laws the allow reordering of 
  arguments of the @mzip@ operation.

\item Unlike the order of arguments, the order of clauses is important. As specified by 
  \textit{clause ordering}, when there are overlapping alternatives, the first one should be 
  selected.

\item The transformations \textit{alternative noninterference} and \textit{argument noninterference}
  specify that including additional failing clause or argument doesn't affect the meaning. (In 
  equation (3), the symbol $\bullet$ stands for a pattern that never succeeds.) The equations are
  manifested as laws that identify @mzero@ as \textit{zero} and \textit{unit} of |mzip_| and
  |morelse_|, respectively.

\item The next three transformations describe the case when arguments are created in some special way.
  In \textit{matching units} and \textit{matching images}, the arguments are created by monadic
  @return@ and @map@ (the latter one is defined in terms of |>>=| and @return@ and is also known
  as @liftM@). The arguments of \textit{matching duplicate} are two copies of the same monadic
  value.

\item The \textit{flattening} transformation specifies that @docase@ nested in an argument of 
  another @docase@ can be flattened. This is a trivial fact about non-monadic @case@, but it 
  implies an interesting relation between the operations of a joinad.
\end{itemize}

The following section discusses individual joinad operations and the laws that they must obey
in order to allow the transformations above.

% ==================================================================================================

\section{Joinad laws}
\label{sec:laws}

Before looking at the laws of joinad operations, we briefly review the well-known laws that are 
required for any monad and therefore also for any joinad:
\begin{align*}
\text{unit} \: a \bind f &\equiv f \: a \tag{left identity} \\
m \bind \Varid{unit} &\equiv m \tag{right identity} \\
(m \bind f) \bind g &\equiv m \bind \lambda x \rightarrow f \: x \bind g \tag{associativity}
\end{align*}
Joinad also requires the @mzero@ operation from \ident{MonadZero}. The value should behave as a 
zero element with respect to binding:
\begin{align*}
\mzero \bind f &\equiv \mzero \tag {left zero} \\
m \bind \lambda \rightarrow \mzero &\equiv \mzero \tag{right zero}
\end{align*}
The \textit{left zero} law is generally accepted. The \textit{right zero} law is sometimes omitted, 
because it may not hold when $m$ is $\bot$, but the official documentation for \ident{MonadPlus}
\cite{monadplusdoc} includes it. All of these five laws are necessary to prove that the 
transformations in Section \ref{sec:reasoning} preserve meaning. In particular, the \textit{right 
zero} law is needed to perform the \textit{alternative noninteference} (4) transformation.

% --------------------------------------------------------------------------------------------------

\subsection{MonadZip type class}
\label{sec:laws-monadzip}

To our understanding, the laws about \ident{MonadZip} in the monad comprehensions extension
\cite{bringbackmc} is still subject to discussion. This section discusses laws that are required
about |mzip_| by joinads. The laws presented here are also a reasonable requirement for monad 
comprehensions. We give more details in \ref{sec:proposals}, but many of the syntactic 
transformations from section \ref{sec:reasoning} can be rewritten using the extended 
\textit{monad comprehension} syntax. 

The first two laws allow arbitrary rearrangement of arguments aggregated using the |mzip_| operation.
This is required by \textit{argument ordering} (2). The laws are expressed using two helper functions:
\begin{align*}
  a \mzip (b \mzip c) &\equiv \map \: \text{assoc} \: ((a \mzip b) \mzip c) \tag{associativity} \\
  a \mzip b &\equiv \map \: \text{swap} \: (a \mzip b) \tag{symmetry} \\
\\[-10pt]
  \textbf{where} & \: \text{assoc}((a,b), c) = (a,(b,c))\\
                 & \: \text{swap}(a,b) = (b,a)
\end{align*}
As discussed in Section \ref{sec:theory-monoidal}, these two laws are founded in mathematical theory 
behind joinads. The |mzip_| operation is component of a \textit{symmetric monoidal functor} that requires 
both of the above laws. Another law that is also required by this formalism is \textit{naturality}, which 
is one of three laws that specify how |mzip_| behaves with respect to other operations:

\begin{align*}
  \map \: f \: a \mzip \map \: g \: b &\equiv \map \: (f \times g) \: (a \mzip b) \tag{naturality} \\
  \text{return} \: a \mzip \text{return} \: b &\equiv \text{return} \: (a, b) \tag{product} \\
  a \mzip a &\equiv \map \: \text{dup} \: a \tag{duplication} \\
\\[-10pt]
  \textbf{where} & \: \text{dup} \: a = (a, a)
\end{align*}
In general, all of these three laws specify what happens when applying $\mzip$ to monadic values 
that are created in some special way. The laws follow from the three \textit{matching} 
transformations (6, 7, 8).
\begin{itemize}
\item In \textit{naturality}, the arguments are created using @map@. The application of $\mzip$ is
  lifted to be performed before the mapping. The mapping then transform both elements of the 
  combined tuple.
  
\item In \textit{product}, the arguments are created using @return@. The combination of values is
  lifted to be performed before the use of @return@ and becomes a tuple constructor.

\item In \textit{duplication}, the arguments are two copies of the same monadic value. The 
  duplication can be lifted inside the monad and performed on the actual values using @map@.

\end{itemize}
The next law specifies that @mzero@ is the \textit{zero element} with respect to $\mzip$
(thanks to the symmetry of |mzip_|, it is both left and right zero):
\begin{align*}
  a \mzip \mzero \equiv \mzero \equiv \mzero \mzip a \tag {zero} 
\end{align*}
This law is intuitively necessary. An @mzero@ value of type |m a| does not contain any value of type
|a|. Given a value of type |b| there is no way to construct a value of type |(a, b)|.

\paragraph{Applicative functors.} Given a monad, it is possible to define an instance of applicative
functor (\ident{Applicative} type class). An equivalent definition of this type class described
by McBride and Paterson \cite{applicative} defines an operation $\star$ that has exactly 
the same type as $\mzip$. However, the laws we require are different, because \ident{Applicative}
does not require \textit{symmetry}. 

This means that the $\mzip$ operation is an independent addition. As discussed in our earlier
paper \cite{joinads}, for commutative monads, the $\mzip$ operation can be defined using |>>=|
and @return@, but this is not possible in general. For some types, the $\mzip$ operation is a 
part of a distinct \ident{Applicative} instance. For example, $\star$ for a standard \ident{List}
monad is a Cartesian product. As the name suggests, $\mzip$ for lists is just @zip@, which is 
a $\star$ operation of the \ident{ZipList} applicative functor.

% --------------------------------------------------------------------------------------------------

\subsection{MonadOr type class}
\label{sec:laws-monador}

The \ident{MonadOr} type class defines the @morelse@ operation (also written as $\morelse$). As
already mentioned in Section \ref{sec:extension-typeclass}, it represents a left-biased monoid.
Similarly to other operations, @morelse@ should obey a form of the naturality law. An operation of 
monoid should be associative and have a unit element. In case of joinads, the unit is @mzero@:
\begin{align*}
  (u \morelse v) \morelse w &\equiv u \morelse (v \morelse w) \tag{associativity} \\
  (\map \: f \: u) \morelse (\map \: f \: v) &\equiv \map \: f \: (u \morelse v) \tag{naturality} \\
  u \morelse \mzero \equiv \: u &\equiv \mzero \morelse u \tag{unit}
\end{align*}
Both of the laws are important for joinads. The \textit{unit} law makes it required by the
\textit{alternative noninterference} (4) transformation. The \textit{associativity} law does not 
directly correspond to any transformation, but it specifies that the bracketing does not matter when
aggregating the alternatives using $\morelse$. The \textit{associativity} law makes this an 
implementation detail of the translation.

The left bias of the |morelse_| operation is required by \textit{clause ordering} (3).
The transformation gives the following law:
\begin{align*}
  u \morelse \map \: f \: u \equiv u \tag{left bias}
\end{align*}
The law considers a monadic value and a value that is created from it using @map@. When choosing 
between the two, the $\morelse$ operation constructs a value that is equivalent to the left 
argument. Intuitively, the @map@ operation creates a monadic value with the same structure as the 
original. The law specifies that $\morelse$ prefers values from the left argument when both 
arguments have the same structure.
The \textit{left bias} law is different than the \textit{left catch} law that is 
required about \ident{MonadOr} in the proposal \cite{monadplusreform}. We return to this topic in 
Section \ref{sec:proposals-morelse}, which discusses proposals for Haskell libraries.

Finally, perhaps the most intriguing law is required by the \textit{flattening} (9) transformation.
The law relates the $\morelse$ operation with $\mzip$ and fits very nicely with the rest of the 
theory: 
\begin{align*}
  a \mzip (b \morelse c) \equiv (a \mzip b) \morelse (a \mzip c) \tag{distributivity}
\end{align*}
In context of joinads, the law describes a situation when we have three monadic values and need to
get a value of the first one together with the second or the third one. In this case, the choice
can be done on the two alternative values or on the two alternatives combined with the always 
required value.
The law specifies only left distributivity, but thanks to the symmetry of $\mzip$, 
it also implies right distributivity.

% --------------------------------------------------------------------------------------------------

\subsection{MonadDuplicate type class}
As discussed in Section \ref{sec:intro-aliasing-parallel}, the initial motivation for adding the 
@mduplicate@ operation is to allow aliasing of computations. Interestingly, the type signature
of the operation corresponds to the \textit{cojoin} operation of a \textit{comonad} (a categorical
dual of monad). We first discuss the laws about @mduplicate@. The relation with comonads is 
discussed later (Section \ref{sec:theory-comonads})

The type signature of @mduplicate@ is |m a -> m (m a)|. The first law that relate it to 
monads can be more easily expressed using the \textit{join} operation, which has a dual type: 
|m (m a) -> m a|. For any monad join can be defined as $\textit{join} \: a = a \bind id$. 
The required transformation \textit{binding equivalence} (1) specifies that the aliasing doesn't 
have any effect when a computation is used just once. This maps to the following law (@mduplicate@
is abbreviated as \textit{mdup}):
\begin{align*}
\text{join} \: (\mduplicate \: a) &\equiv a \tag{cancellation}
\end{align*}
In this law, applying @mduplicate@ to a value of type |m a| yields a value of type |m (m a)|. 
Applying \textit{join} to the result should give the same |m a| value. We could also consider the 
composition in the opposite order, but that is too restrictive. By flattening the type, 
\textit{join} arguably removes some information about the structure, which cannot be, in general,
recovered by @mduplicate@.

The next two laws about @mduplicate@ are needed by the \textit{flattening}  (9) transformation. They
specify that we can flatten nested applications of @mduplicate@ and that the |mzip_| operation
can be lifted over @mduplicate@:
\begin{align*}
(\mduplicate \: m \bind f) \mzip n &\equiv \mduplicate \: m \bind (\mzip \: n) \circ f  \tag{lifting}\\
\mduplicate \: (\mduplicate \: m \bind k) \bind l &\equiv \mduplicate \: m \bind l \circ k   \tag{nesting}
\end{align*}
Finally, we also require laws that specify how @mduplicate@ behaves with respect to |>>=|. The three 
special cases cover @mzero@, @return@ and @mzip@. They are required by \textit{argument noninterference} 
(5), \textit{matching units} (6) and \textit{matching images} (7):
\begin{align*}
\mduplicate \: \mzero \bind f &\equiv f \: \mzero \tag{zero}\\
\mduplicate \: (\text{return} \: a) \bind f &\equiv f \: (\text{return} \: a) \tag{unit} \\
\mduplicate \: (\map \: f \: m) \bind g &\equiv \mduplicate \: m \bind g \circ \map \: f \tag{naturality}
\end{align*}
The section \ref{sec:extension-typeclass} includes a default implementation of @mduplicate@ using 
the @return@ operation of a monad. It is easy to see that this trivial definition obeys the above laws. 
Using @return@ to define @mduplicate@ is always valid, but it isn't 
always useful. In particular, aliasing was essential for the parallelism joinad.

% ==================================================================================================

\section{Theory of joinads}
\label{sec:theory}

This section expands on some ideas discussed from the previous section. We look at categorical 
foundations of joinads and consider them as an algebraic structure.

% --------------------------------------------------------------------------------------------------

\subsection{Monoidal functors}
\label{sec:theory-monoidal}

The discussion about \ident{MonadZip} and \ident{Applicative} from Section \ref{sec:laws-monadzip}
can be recast in terms of category theory. Haskell's monad is a monad in category theory and 
\ident{Applicative} is a monoidal functor. Given a monad, we can construct a monoidal functor. The 
\ident{MonadZip} type class with the laws given above corresponds to a \textit{symmetric monoidal 
functor} and $\mzip$ is the natural transformation defined by it. This is another justification for 
the \textit{naturality}, \textit{associativity} and \textit{symmetry} laws. 

Joinads combine this symmetric monoidal functor with a monad and thus also a monoidal functor 
specified by the monad. The underlying functor is the same, but the natural transformation and units 
differ. The unit of the $\mzip$ operation is not needed by joinads, so we do not require users to 
define it. In particular, this means that @return@ does not behave as unit with respect to $\mzip$. 
For example, a unit for \ident{List} is a singleton list, but unit for \ident{ZipList} is an infinite
list. 

Zipping a list with an infinite list and then projecting out first elements of a tuple
gives the original list, but the same isn't true for zipping with a singleton list.

% --------------------------------------------------------------------------------------------------

\subsection{Comonads}
\label{sec:theory-comonads}

\begin{figure}
\begin{code}
class Functor m => Comonad m where
  cojoin :: m a -> m (m a)
  coreturn :: m a -> a
\end{code}
\caption{Definition of \ident{Comonad} type class. }
\label{fig:comonad-class}
\end{figure}

A comonad is a dual of monad. One way to define a comonad in Haskell is shown in 
\ref{fig:comonad-class}. The definition extends \ident{Functor} with two additional operations. 
A monad can be defined similarly by extending \ident{Functor} with dual operations.
Although less frequent than monads, comonads are also a useful notion of computations in 
functional programming \cite{comomads-ypnos, comonads-dataflow}.

The @cojoin@ operation of a comonad could be used as the basis of @mduplicate@. Joinads do not 
need the rest of the comonadic structure (the @coreturn@ operation). The @mzip@ operation is 
defined similarly---it arises from some (symmetric) monoidal functor, but the rest of the structure 
(the unit) is not needed. 

Determining whether every @mduplicate@ operation should be defined by some comonad is a future
work. However, we can partly explore it by looking at the comonad laws:
\begin{align*}
\map \: \text{cojoin} \: (\text{cojoin} \: a) &\equiv \text{cojoin} (\text{cojoin} \: a)\\
\map \: (\map \: f) \: (\text{cojoin} \: a) &\equiv \text{cojoin} \: (\map \: f \: a)\\
f (\text{coreturn} \: a) &\equiv \text{coreturn} \: (\map \: f \: a)\\
\map \: \text{coreturn} \: (\text{cojoin} \: a) &\equiv \text{coreturn} \: (\text{cojoin} \: a) \equiv \text{id}
\end{align*}
The first two laws are defined just in terms of @map@ and @cojoin@, so they could be also required 
by joinads. We did not include them, because they are not directly implied by any syntactic 
transformations form section \ref{sec:reasoning}. The last two laws specify the behaviour of 
@coreturn@, so they are not directly relevant to joinads.

The discussion in the previous paragraph rises an interesting question from the theoretical 
perspective. Monads and comonads are both useful notions of computations, but what if we work
with computations that are both monadic and comonadic?

Every monad arises from an adjunction. Given two functors adjoint functors 
$F \dashv U$ with unit $\eta$ and counit $\epsilon$ where $F : \mathcal{C} \rightarrow \mathcal{D}$ 
and $U : \mathcal{D} \rightarrow \mathcal{C}$, it is possible to define a monad $(T, \eta, \mu)$
by taking $T = U \circ F$ and $\mu = U \eta F$. Interestingly, we can also define a comonad 
$(S, \epsilon, \delta)$ by taking $S = F \circ U$ and $\delta = F \mu U$.

A notion of computation that is both monadic and comonadic should have just a single functor,
which means that we would require that $T \equiv S$. This implies that the functors $F$ and $U$ must 
have the same domain and codomain. In particular, the two adjoint functors must be both definable
in Haskell. Finally, the \textit{cancellation} law adds a requirement specifying that 
$\mu \circ \delta = U \eta F^2 \mu U\equiv \textit{id}$. We consider this an interesting area
for future work.

% --------------------------------------------------------------------------------------------------

\subsection{Joinad algebra} 
\label{sec:theory-algebra}

The laws about |mzip_| and |morelse_| discussed in the previous section suggest that 
joinads can be modelled as an algebraic structure. The @mzero@ value behaves as \textit{zero}
with respect to |mzip_| and as \textit{unit} with respect to |morelse_|. In addition, the
|mzip_| operation distributes over |morelse_|. An algebraic structure that satisfies the 
distributivity law is called a \textit{ringoid}.

When given a set $J$ containing all monadic values of some type |m a|, we can view joinad as a 
ringoid $(J, \morelse, \mzip)$. The @mzero@ value is a special element $0 \in J$.
The structure $(J, \mzip)$ is a commutative semigroup with $0$ as the zero element and the
structure $(J, \morelse)$ is a monoid with $0$ as the unit. This specification captures the 
essence of joinads---the only thing that is left out is the left bias of the $\morelse$ 
operation. As discussed in section \ref{sec:related-monadplus} this general case may be also 
useful, but it doesn't fully capture the semantics inspired by the Haskell's @case@ construct.

% ==================================================================================================

\section{Feature interactions and library proposals}
\label{sec:proposals}

As already mentioned, joinads combine several type classes that are already known to the Haskell 
community. This section discusses how joinads relate to other work based on the same type classes. 
We also propose how to adjust the laws required by these type classes to better capture their 
essence.

% --------------------------------------------------------------------------------------------------

\subsection{Parallel monad comprehensions}
\label{sec:proposals-monadzip}

Joinads use the @mzip@ operation of the \ident{MonadZip} type class to express parallel composition 
of monadic computations. The operation is used when an alternative of @docase@ consists of multiple 
binding patterns. The type class has been as part of \textit{parallel monad comprehensions} 
\cite{bringbackmc, comprefun} extension in GHC.

Monad comprehensions \cite{comprehendingmonads} are a generalization of list comprehensions for 
working with arbitrary monads. A recent reimplementation of the idea generalizes additions to the 
list comprehension language including grouping and ordering \cite{groupordercompre}, as well as 
parallel list comprehensions. A parallel list comprehension contains multiple independent generators 
separated by a bar:

\begin{code}
[ (i, c) | c <- ``Hello world!'' | i <- [ 0 .. ] ]
\end{code}
The syntax corresponds to using the @zip@ function, so the snippet creates a list of 
pairs containing characters with their corresponding indices. A monadic version generalizes @zip@
to the @mzip@ function and embeds it into a \ident{MonadZip} type class. The generalized syntax 
can be used to encode parallel composition of parsers from the Section \ref{sec:intro-combining-parsers}
(more details can be found in our article \cite{parcomprefun}):

\begin{code}
validPhone = 
  [ num  | num  <- many (sat isDigit) 
         | _    <- replicateM 10 item
         | _    <- startsWith (string "1223") ]
\end{code}
The desugaring of this snippet is exactly the same as the initial desugaring of @docase@ presented in 
Section \ref{sec:intro-combining-parsers}. This suggests that a \textit{parallel monad comprehension}
expression is equivalent to a @docase@ expression which has generator expressions as arguments and 
consists a single alternative with body containing @return@. The actual desugaring of joinads is more 
complicated as it adds aliasing and wrapping to support commit points. However, these additional 
features do not have any effect in this simple case.

\subsection{Monad comprehension laws} 
At the time of writing, the extension adding parallel monad comprehensions and 
\ident{MonadZip} required the following laws:
\begin{align*}
  \map \: f \: a \mzip \map \: g \: b &\equiv \map \: (f \times g) \: (a \mzip b) \tag{naturality} \\
  \map \: \text{fst} \: (a \mzip b) &\equiv a \tag{information preservation} 
\end{align*}

\paragraph{Monoidal laws.}
The \textit{naturality} law was also proposed for joinads (Section \ref{sec:laws-monadzip}). It
arises from the joinad laws, but also from the fact that the @mzip@ operation is defined by a 
monoidal functor. 

\begin{itemize}
\item We propose adding \textit{associativity} law. The law also arises from a monoidal functor,
  therefore it seems very reasonable to require the two laws together.
  
\item We believe that symmetry is an essential aspect of @mzip@ and we argue that 
  the \textit{symmetry} law should be included.

\end{itemize}
The symmetry of @mzip@ holds for lists as well as for the \ident{MonadZip} instances presented in 
this paper. In terms of parallel monad comprehensions, the law guarantees the following 
equivalence:

\begin{code}
[(a, b) | a <- m^sub(1) | b <- m^sub(2)] == [(a, b) | a <- m^sub(2) | b <- m^sub(1)]
\end{code}
The symmetry law means that the @mzip@ operation cannot be automatically implemented in 
terms of |>>=| and @return@. This is a useful requirement---it says that the additional syntax
should also have an additional meaning. It is still possible to get @mzip@ for free, 
but only for \textit{commutative monads}. For some monads, such as the @Reader@ monad, parallel 
composition doesn't add any additional behaviour. More information about the relations with 
commutative monads can be found in our previous work \cite{joinads, eventsthesis}.

\paragraph{Relation with map.}
The \textit{information preservation} law specifies that we can recover the original monadic values 
from a monadic value created using |mzip_|. This law is problematic. It doesn't hold for lists |a| 
and |b| of different length. The @zip@ function restricts the length of the result to the length 
of the shorter list.

The problem is that the law allows applying @mzip@ to inputs with different structure (in case of 
lists, the length of the list), but recovering the original values is only possible if their 
structure is the same. Using \textit{naturality} and the \textit{duplication} law, we can derive the
following law that looks similar and clarifies the requirement about the structure of values:

\begin{align*}
  \map \: \text{fst} \: (a \mzip \: \map f a) \equiv a \equiv \map \: \text{snd} \: (\map \: g \: a \: \mzip \: a)
\end{align*}
Instead of zipping two arbitrary monadic values, the law zips a value with an image created using 
@map@ (using some complete functions |f| and |g|). Thanks to the properties of @map@,
the law only concerns zipping of monadic values with the same structure.

\begin{itemize}
\item The \textit{information preservation} law does not hold for many standard implementations of 
  @mzip@, so we propose replacing it with the weaker form or with equivalent \textit{duplication} law.

\item In terms of monad comprehensions, the \textit{product} law of joinads states that 
  |[(ma, mb)|| ma <- return a || mb <- return b]| should be equivalent to |return (a, b)|. We find 
  this a useful requirement for monad comprehensions.
\end{itemize}

% --------------------------------------------------------------------------------------------------

\subsection{Left-biased additive monads}
\label{sec:proposals-morelse}

The class of monads that provide a special zero value of type |m a| and an operation of type 
|mplus_ :: m a -> m a -> m a| can be captured using the \ident{MonadPlus} type class. It is widely 
accepted that such a monad should form a monoid:

\begin{align*}
  \mzero \mplus u &\equiv u \equiv u \mplus \mzero \tag{monoid identity}\\
  (u \mplus v) \mplus w &\equiv u \mplus (v \mplus w) \tag{monoid associativity}
\end{align*}
However, there is some disagreement about the additional laws. The \ident{MonadPlus} reform proposal
\cite{monadplusreform} provides a solution by splitting the type class into \ident{MonadPlus} 
obeying \textit{left distribution} law and \ident{MonadOr} obeying \textit{left catch} law. The
\textit{left bias} law that we require for joinads (Section \ref{sec:laws-monador}) adds a third
alternative:

\begin{align*}
  u \mplus v \bind f &\equiv (u \bind f) \mplus (v \bind f) \tag{left distribution}\\
  (\text{return} \: a) \mplus u &\equiv \text{return} \: a \tag{left catch} \\
  u \morelse \map \: f \: u &\equiv u \tag{left bias}
\end{align*}
It is not difficult to find counter-examples showing that none of the three laws implies some other.
Both \textit{left bias} and \textit{left catch} represent some form of left bias, but in a 
different way. 

\begin{itemize}
\item The \textit{left bias} law uses an arbitrary value as the left
  and a special value (constructed using @map@) as the right argument. 

\item The \textit{left catch} law uses an arbitrary value as the right 
  and a special value (constructed using @unit@) as the left argument. 
\end{itemize}
Despite the difference, the main purpose of the two laws is the same. They both specify that the 
operation is left biased. Which law should hold about \ident{MonadOr}? One option is to consider 
upper or lower bound of the two laws:

\begin{align*}
  (\text{return} \: a) \mplus (\text{return} \: b) &\equiv \text{return} \: a \tag{lower bound} \\
  u \morelse u \bind f &\equiv u \tag{upper bound}
\end{align*}
The \textit{upper bound} implies by both left bias and left catch, while the \textit{lower
bound} is implied by any of the two. It is not clear to us whether any monad can provide a 
non-trivial implementation of |mplus_| satisfying the \textit{upper bound} law. 
The \textit{lower bound} law is more appropriate, although it isn't sufficient to prove that the 
\textit{clause ordering} equation from Section \ref{sec:reasoning} holds.

Another option is to accept \textit{left bias} as a law of \ident{MonadOr}. The most prominent 
monad that implements \ident{MonadOr} is the \ident{Maybe} monad, which obeys both of the laws. In
addition, we can give a useful implementation of @morelse@ that obeys the left bias law for 
the \ident{List} monad. The two declarations are shown in Figure \ref{fig:monador-instances}.

\begin{figure}
\label{fig:monador-instances}
\begin{code}
instance MonadOr List where
  morelse (x:xs) (y:ys) = x:(morelse xs ys)
  morelse [] ys = ys
  morelse xs [] = xs

instance MonadOr Maybe where
  morelse (Just a) _ = (Just a)
  morelse Nothing b = b
\end{code}
\caption{Instance of \ident{MonadOr} that obey left bias law}
\end{figure}

% ==================================================================================================

\section{Related and future work}
\label{sec:related}

We first presented an earlier version joinads in the \fsharp language \cite{joinads} using 
different examples that also included embedding of join calculus. An article for
The Monad.Reader \cite{parcomprefun} provides more details on the relation between joinads
and \textit{monad comprehensions}.

The rest of this section presents some of the important related work on pattern matching, concurrent
programming and abstract computation types. We also briefly introduce preliminary ideas that 
could be interesting topic for future work.

% --------------------------------------------------------------------------------------------------

\subsection{Backtracking and committing patterns}
\label{sec:related-monadplus}

The existing work on pattern matching has focused on enabling pattern matching on abstract values 
using views \cite{views-haskell, views-sml}. A similar concept also appeared in \fsharp and Scala 
\cite{scala-patternmatching, activepatterns}. Making patterns first-class made it possible to encode 
Join calculus in Scala \cite{scala-encodingjoins}, although the encoding is somewhat opaque.

Several authors \cite{firstlcasspats, activepatterns} have suggested generalizing the result of 
pattern matching from \ident{Maybe} (representing a failure or a success) to any additive monad 
using the \ident{MonadPlus} type class. The concrete examples included encoding of backtracking 
using the \ident{List} monad and composing transactions using \ident{STM}.

The following example uses parsers to demonstrates the difference between joinads and the 
interpretation based on \ident{MonadPlus}. Assuming standard set of parser combinators, we can write:

\begin{code}
body = mcase_ char '(', many item of 
  _, ? -> do  str <- body
              _ <- char ')'
              return str
  ?, str -> return str
\end{code}
The @mcase@ construct (similar to our @docase@) represents a monadic pattern matching using 
\ident{MonadPlus}. In the syntax designed for active patterns \cite{activepatterns}, monadic 
values were produced by \textit{active patterns} that produced monadic values. For parsers, the 
type of active patterns would be |a -> Parser b|. This leads to a different syntax, but it is 
possible to translate between the two options.

The proposed translation using \ident{MonadPlus} follows similar pattern as the translation of joinads,
but differs in three ways:

\begin{code}
(char '(' >>= \_ -> body >>= \str ->
    char ')' >>= \_ -> return str) mplus_
(many item >>= \str -> return str)
\end{code}
The first difference (apparent from this example) is that the proposed encoding using \ident{MonadPlus}
does not add additional wrapping around the body of the alternatives to support committing to an 
alternative. The second difference is that \ident{MonadPlus} usually requires the \textit{left
distributivity} law instead of the \textit{left bias} law required by \ident{MonadOr}. Finally, the 
third difference (not demonstrated by the snippet) is that multiple binding patterns are translated 
using nested |>>=| instead of a special operation provided by \ident{MonadZip}. 

\begin{itemize}
\item When using the \ident{MonadOr} laws, the |mplus_| operation attempts to parse the input using the
  first alternative. Other alternatives are considered only if the first one fails. The parser from 
  the example above would deterministically parse ``((1))'' as ``1''. The laws of \ident{MonadPlus} 
  make the resulting parser non-deterministic and so it parses the same input as one of the following
  three options: ``1'', ``(1)'' or ``((1))''.

\item Without the additional wrapping, the parser needs to implement backtracking. If the
  input is ``(1'', the first alternative is selected, but it fails, meaning that the left argument of 
  |mplus_| will be a failure. Therefore the parser needs to backtrack and try the second alternative. 
  Adding the wrapping means that the parser will commit to the first alternative. Running the parser
  specified by the body eventually fails, so the result will be a failure.

\item The fact that joinads use a special @mzip@ operation for combining multiple inputs means that 
  we can require a symmetric operation even for non-commutative monads. Additionally, using a 
  separate @mzip@ operation may enable additional optimizations, for example, in the \ident{STM} monad.

\end{itemize}
The example above shows that all of the options may have a feasible meaning for some monads. The 
version based on joinads seems more appropriate for modern monadic parsers such as Parsec 
\cite{parsec}, which provides left-biased @monador@ as the @<|>@ combinator.

In conclusion, we find the joinad-based semantics of @docase@ that supports commit points more 
appropriate for monads from functional programming. The variant using \ident{MonadPlus} often 
implies backtracking and thus may be more suitable for logic programming languages such as Prolog. 

% --------------------------------------------------------------------------------------------------

\subsection{Commit points in remote procedure calls}
The discussion about commit points in Section \ref{sec:intro-commit-poorman} was inspired by 
Concurrent ML (CML) \cite{concurrentml}. CML is a concurrent programming language built on top
of Standard ML. It supports first-class synchronization values called \textit{events} that can 
be used to encode many common concurrent programming patterns.

Joinads, on the other hand, capture a single pattern that we find extremely important. To 
demonstrate the relation between joinads and CML, we consider two options for implementing a 
remote procedure call (RPC). Assume we have a monad for synchronous (blocking) communication
and monadic values @send@, @recv@ that represent sending request to a server and receiving
a response from the server. As an alternative to performing the RPC call, the client can choose
to perform @other@ operation.

One way to implement the RPC call is to initiate a call, but allow abandoning the RPC communication
at any time until it completes. This means that receiving a response from the server is used
as a commit point for the RPC call:

\begin{code}
docase_ send, recv, other of
  (), res, ?  -> handleRpc res
  ?, ?, o     -> handleOther
\end{code}
We can assume that the event @recv@ becomes enabled after @send@, so the first alternative becomes
enabled after the server replies. The second alternative will be selected if the @other@ event is
enabled earlier. This may, or may not, be before the server accepts the request and @send@ event
becomes enabled.

The second way to implement a RPC call is to allow abandoning the RPC communication only before
the server accepts the communication request. After that, the client needs to wait for @recv@ event
and cannot choose @other@ instead:

\begin{code}
docase_ send, other of
  (), ?  ->  do res <- recv
             handleRpc res
  ?, o   ->  handleOther
\end{code}
In this version of code, the @docase@ construct only chooses between @send@ and @other@. Once the
first alternative is selected, it has to wait for the server response using @recv@.

This section demonstrates that joinads can capture the two essential patterns for writing RPC
communication as introduced in Concurrent ML. This example critically relies on the support for
commit points introduced in Section \ref{sec:intro-commit-poorman}. When using the simple encoding
discussed in Section \ref{sec:related-monadplus}, the two expressions would translate to the 
same meaning.

% --------------------------------------------------------------------------------------------------

\subsection{Joinads and other computation types}
Joinads extend monads to support the @docase@ construct, but functional languages use several other
notions of computations. In the future, it may be interesting to consider how other computations
relate to generalized pattern matching. Comonads (a categorical dual of monads) \cite{comonads-codata} 
have been used for encoding data-flow programs \cite{comonads-dataflow}, but also for stencil 
computations using special \textit{grid patterns} \cite{comomads-ypnos}. Arrows 
\cite{generalisingmonads, causalarr} are used mainly in functional reactive programming research 
\cite{arrows-frp} and can be written using the arrow notation \cite{arrows-notation} (in a similar 
way monads use the do notation).

Another notion of computation is called \textit{applicative functors} \cite{applicative} or 
\textit{idioms}, which are weaker than monads and can thus capture larger number of computations. 
Haskell libraries also include an \ident{Alternative} type class that extends applicative functors 
with a |diamond_| operator similar to |mplus_| from \ident{MonadPlus} or \ident{MonadOr}. The 
declaration is shown in Figure \ref{fig:alternative-typeclass}. To make the following explanation 
easier, the figure uses \ident{Monoidal} type class, which is equivalent to the more common 
\ident{Applicative} class (as discussed by McBride and Patterson \cite{applicative} in Section 7). 

\begin{figure}
\begin{code}
class Functor f => Monoidal f where
  unit :: f ()
  (star_) :: f a -> f b -> f (a, b)

class Monoidal f => Alternative f where
  empty :: f a
  (diamond_) :: f a -> f a -> f a
\end{code}
\label{fig:alternative-typeclass}
\caption{\ident{Alternative} type class}
\end{figure}

Interestingly, the operations of \ident{Alternative} have the same types as the two most essential 
operations of joinads. The |star_| operation has the same type as our @mzip@, representing parallel 
composition, and |diamond_| has the type of @morelse@, representing a choice. This suggests that 
\ident{Alternative} may define a weaker variant of joinads in a similar way in which applicative 
functors are weaker than monads.

% --------------------------------------------------------------------------------------------------

\subsection{Applications}
We demonstrated that the @docase@ language extension can be used for working with many common 
monads. When using monadic parser combinators \cite{monadparsing} the @morelse@ operation
represents a left-biased choice as supported in \cite{parsec}. As discussed in our earlier article,
our implementation of parallel composition (the @mzip@ operation) corresponds to the intersection
of context-free grammars \cite{parcomprefun}. We are not aware of any parser combinator library that
provides this operation, but it seems to be very useful for validation of inputs (finding inputs
that do not match any of the given parsers).

The reactive programming examples used in this paper were based on imperative streams developed by 
Scholz \cite{imperative-streams}. Imperative streams are essentially monads for synchronous 
reactive programming. A Push-Pull Functional Reactive Programming framework developed by Elliott 
\cite{push-pull-frp} includes a monad instance for events, so it could likely benefit from the
@docase@ syntax too.

The parallel programming model that we presented can be added to various existing Haskell frameworks.
Our earlier article \cite{parcomprefun} used strategies \cite{strategies-new}. In this paper,
we embedded the examples in the \ident{Par} monad \cite{parmonad} with several extensions that 
allow writing speculative computations \cite{parmonad-cancellation}. The programming model is very
similar to the @pcase@ construct provided by Manticore \cite{manticore}.

% ==================================================================================================

\section{Conclusions}
This paper presented a characterization of monadic computations that provide two 
additional operations: \textit{parallel composition} and \textit{choice}. 
These two operations are not new to Haskell. In fact, they are already captured by \ident{MonadZip} 
and \ident{MonadPlus} type classes. We combined these two and an additional operation that captures
\textit{aliasing} of computations and we designed a @docase@ notation that makes it easy to compose
computations using these operations.

The @docase@ notation is inspired by our previous work on joinads in \fsharp. However, this paper
uses a simpler operations that are amenable to formal reasoning. We started with a set of semantic preserving
transformations that are intuitively expected to hold about the @docase@ construct and derived a set
of laws about primitive operations that define joinads as an algebraic structure and show how some 
of the operations related to category theory.

Finally, we also made several concrete library proposals based on our work. In particular, we support
the proposal to distinguish between unbiased \ident{MonadPlus} and left-biased \ident{MonadOr} and we
propose a refined set of laws that should hold about \ident{MonadZip}. We demonstrated the
usefulness of our extension using a wide range of monads including reactive and parallel programming
as well as input validation using monads. 

% ==================================================================================================

\acks
Acknowledgments (to be added).

% ==================================================================================================

\begin{thebibliography}{37}
\raggedright
\providecommand{\natexlab}[1]{#1}
\providecommand{\url}[1]{\texttt{#1}}
\expandafter\ifx\csname urlstyle\endcsname\relax
  \providecommand{\doi}[1]{doi: #1}\else
  \providecommand{\doi}{doi: \begingroup \urlstyle{rm}\Url}\fi

\bibitem[bri()]{bringbackmc}
Bring back monad comprehensions.
\url{http://tinyurl.com/ghc4370}.

\bibitem[mon()]{monadplusdoc}
Control.monoad, haskell documentation.

\bibitem[Brown(2008)]{chp-monad}
N.~C.~C. Brown.
\newblock {C}ommunicating {H}askell {P}rocesses: {C}omposable {E}xplicit
  {C}oncurrency using {M}onads.
\newblock In \emph{{C}ommunicating {P}rocess {A}rchitectures 2008}, pages
  67--83, 2008.

\bibitem[Claessen(1999)]{poorman}
K.~Claessen.
\newblock A poor man's concurrency monad.
\newblock \emph{Journal of Functional Programming}, 9:\penalty0 313--323, May
  1999.
\newblock ISSN 0956-7968.

\bibitem[Elliott(2009)]{push-pull-frp}
C.~Elliott.
\newblock Push-pull functional reactive programming.
\newblock In \emph{Haskell Symposium}, 2009.
\newblock \url{http://conal.net/papers/push-pull-frp}.

\bibitem[Emir et~al.(2007)Emir, Odersky, and Williams]{scala-patternmatching}
B.~Emir, M.~Odersky, and J.~Williams.
\newblock Matching objects with patterns.
\newblock In \emph{ECOOP 2007}, 2007.

\bibitem[Fluet et~al.(2010)Fluet, Rainey, Reppy, and Shaw]{manticore}
M.~Fluet, M.~Rainey, J.~Reppy, and A.~Shaw.
\newblock Implicitly threaded parallelism in manticore.
\newblock \emph{Journal of Functional Programming}, 20\penalty0 (Special Issue
  5-6):\penalty0 537--576, 2010.

\bibitem[Haller and Van~Cutsem(2008)]{scala-encodingjoins}
P.~Haller and T.~Van~Cutsem.
\newblock Implementing {J}oins using {E}xtensible {P}attern {M}atching.
\newblock In \emph{Proceedings of the 10th {I}nternational {C}onference on
  {C}oordination {M}odels and {L}anguages}, 2008.

\bibitem[HaskellWiki(2011)]{monadplusreform}
HaskellWiki.
\newblock Monadplus reform proposal, 2011.
\newblock \url{http://tinyurl.com/monadplus-reform-proposal}.

\bibitem[Hudak et~al.(2003)Hudak, Courtney, Nilsson, and Peterson]{arrows-frp}
P.~Hudak, A.~Courtney, H.~Nilsson, and J.~Peterson.
\newblock Arrows, robots, and functional reactive programming.
\newblock In \emph{Advanced Functional Programming}, volume 2638 of
  \emph{LNCS}, pages 1949--1949. 2003.

\bibitem[Hughes(1998)]{generalisingmonads}
J.~Hughes.
\newblock Generalising monads to arrows.
\newblock \emph{Science of Computer Programming}, 37:\penalty0 67--111, 1998.

\bibitem[Hutton and Meijer(1998)]{monadparsing}
G.~Hutton and E.~Meijer.
\newblock {Monadic Parsing in Haskell}.
\newblock \emph{Journal of Functional Programming}, 8\penalty0 (4):\penalty0
  437--444, July 1998.

\bibitem[Jones and Wadler(2007)]{groupordercompre}
S.~P. Jones and P.~Wadler.
\newblock Comprehensive comprehensions.
\newblock In \emph{Haskell'07}, pages 61--72, 2007.

\bibitem[Kieburtz()]{comonads-codata}
R.~B. Kieburtz.
\newblock Codata and comonads in haskell.

\bibitem[Kmett(2011)]{comonadpkg}
E.~A. Kmett.
\newblock The comonad package, 2011.
\newblock \url{http://hackage.haskell.org/package/comonad}.

\bibitem[Leijen and Meijer(2001)]{parsec}
D.~Leijen and E.~Meijer.
\newblock Parsec: Direct style monadic parser combinators for the real world.
\newblock Technical Report UU-CS-2001-27, Department of Computer Science,
  Universiteit Utrecht, 2001.

\bibitem[Liu et~al.(2009)Liu, Cheng, and Hudak]{causalarr}
H.~Liu, E.~Cheng, and P.~Hudak.
\newblock Causal commutative arrows and their optimization.
\newblock In \emph{ICFP'09}, pages 35--46, 2009.

\bibitem[Marlow et~al.(2010)Marlow, Maier, Loidl, Aswad, and
  Trinder]{strategies-new}
S.~Marlow, P.~Maier, H.-W. Loidl, M.~K. Aswad, and P.~Trinder.
\newblock Seq no more: better strategies for parallel haskell.
\newblock In \emph{Haskell Symposium}, 2010.

\bibitem[McBride and Paterson(2007)]{applicative}
C.~McBride and R.~Paterson.
\newblock Applicative programming with effects.
\newblock \emph{Journal of Functional Programming}, 18:\penalty0 1--13, 2007.

\bibitem[Okasaki(1998)]{views-sml}
C.~Okasaki.
\newblock Views for standard ml.
\newblock In \emph{ML Workshop}, 1998.

\bibitem[Orchard et~al.(2010)Orchard, Bolingbroke, and Mycroft]{comomads-ypnos}
D.~A. Orchard, M.~Bolingbroke, and A.~Mycroft.
\newblock Ypnos: declarative, parallel structured grid programming.
\newblock DAMP '10, 2010.

\bibitem[Paterson(2001)]{arrows-notation}
R.~Paterson.
\newblock A new notation for arrows.
\newblock In \emph{International Conference on Functional Programming}, pages
  229--240. ACM Press, Sept. 2001.

\bibitem[Petricek(2010)]{eventsthesis}
T.~Petricek.
\newblock Reactive programming with events.
\newblock Master's thesis, Charles University in Prague, 2010.

\bibitem[Petricek(2011{\natexlab{a}})]{parcomprefun}
T.~Petricek.
\newblock Fun with parallel monad comprehensions, 2011{\natexlab{a}}.
\newblock \url{http://tbd.net}.

\bibitem[Petricek(2011{\natexlab{b}})]{parmonad-cancellation}
T.~Petricek.
\newblock Explicit speculative parallelism for haskell's par monad,
  2011{\natexlab{b}}.
\newblock \url{http://tomasp.net/blog/speculative-par-monad.aspx}.

\bibitem[Petricek and Syme(2011)]{joinads}
T.~Petricek and D.~Syme.
\newblock Joinads: A retargetable control-flow construct for reactive, parallel
  and concurrent programming.
\newblock In \emph{PADL'11}, pages 205--219, 2011.

\bibitem[Reppy(2007)]{concurrentml}
J.~H. Reppy.
\newblock \emph{Concurrent programming in ML}.
\newblock Cambridge University Press, 2007.
\newblock ISBN 978-0-521-71472-3.

\bibitem[Scholz(1998)]{imperative-streams}
E.~Scholz.
\newblock Imperative streams-a monadic combinator library for synchronous
  programming.
\newblock ICFP, 1998.

\bibitem[Schweinsberg(2010)]{comprefun}
N.~Schweinsberg.
\newblock Fun with monad comprehensions, 2010.
\newblock \url{http://tinyurl.com/comprehension-fun}.

\bibitem[Simon~Marlow(2011)]{parmonad}
S.~P.~J. Simon~Marlow, Ryan~Newton.
\newblock A monad for deterministic parallelism, 2011.
\newblock \url{http://tinyurl.com/monad-par}.

\bibitem[Swierstra(2008)]{parsingtutorial}
S.~D. Swierstra.
\newblock Combinator parsing: A short tutorial.
\newblock Technical report, Utrecht University, 2008.
\newblock \url{http://tinyurl.com/parsing-tutorial}.

\bibitem[Syme et~al.(2007)Syme, Neverov, and Margetson]{activepatterns}
D.~Syme, G.~Neverov, and J.~Margetson.
\newblock Extensible pattern matching via a lightweight language extension.
\newblock ICFP, 2007.

\bibitem[Trinder et~al.(1998)Trinder, Hammond, Loidl, and {Peyton
  Jones}]{strategies-old}
P.~W. Trinder, K.~Hammond, H.-W. Loidl, and S.~L. {Peyton Jones}.
\newblock Algorithm + {S}trategy = {P}arallelism.
\newblock \emph{Journal of Functional Programming}, 8\penalty0 (1):\penalty0
  23--60, Jan. 1998.

\bibitem[Tullsen(2000)]{firstlcasspats}
M.~Tullsen.
\newblock First class patterns.
\newblock In \emph{PADL 2000}, 2000.

\bibitem[Uustalu and Vene(2005)]{comonads-dataflow}
T.~Uustalu and V.~Vene.
\newblock The essence of dataflow programming.
\newblock In \emph{APLAS}, pages 2--18, 2005.

\bibitem[Wadler(1987)]{views-haskell}
P.~Wadler.
\newblock Views: a way for pattern matching to cohabit with data abstraction.
\newblock POPL, 1987.

\bibitem[Wadler(1990)]{comprehendingmonads}
P.~Wadler.
\newblock Comprehending monads.
\newblock In \emph{Proceedings of the 1990 ACM conference on LISP and
  functional programming}, pages 61--78, 1990

\end{thebibliography}

% ==================================================================================================

\appendix

\section{Explicit shortcircuiting}
\label{sec:appendix-shortcircuit}
The motivating example in \ref{sec:motivation} used @docase@ and the \ident{Par} monad to implement
a @forall@ function for trees. The function takes a predicate and tests whether it holds for any
value stored in any leaf of the tree. The function should process both subtrees in parallel and
it should implement shortcircuiting behaviour. This means that it should immediately return when
the predicate does not hold for some value. The following listing implements the same functionality
directly (using an extension that allows cancellation of tasks \cite{parmonad-cancellation}).

\begin{code}
forall :: (a -> Bool) -> Tree a -> Par Bool

forall p tree = do
    tok <- newCancelToken
    r <- forall' tok tree
    cancel tok 
    return r
  where 
    forall' tok (Leaf v) = return (p v)
    forall' tok (Node left right) = do
      leftRes <- new
      rightRes <- new
      finalRes <- newBlocking
      forkWith tok (forall' tok left >>= 
        completed leftRes rightRes finalRes)
      forkWith tok (forall' tok right >>= 
        completed rightRes leftRes finalRes)
      get finalRes
    
    completed varA varB fin resA = do
      put varA resA
      if not resA then put fin False
      else get varB >>= put fin . (&& resA)
\end{code}

The main body of the function creates a new cancellation token using @newCancelToken@ and then 
calls a helper that does the actual processing. The cancellation token is used when starting any 
background computations during the recursive processing. As a result, when the helper function returns 
@False@ (meaning that some computations may be still running), we can cancel all (still running, but 
unnecessary) computations using this single token.

The function that implements the recursive processing first creates three variables. The first two 
(@leftRes@ and @rightRes@) are used by their corresponding computations (when a left computation completes, 
it writes result to @leftRes@). The last variable is created using the @newBlocking@, which creates a
variable that blocks computations that attempt to write into it when it is full. The function then spawns 
two tasks to process sub-trees and waits for the final result.

The two background computations spawned by the function both make a recursive call and then pass the 
result to the completed function. It takes all three variables (variable of the currently completed 
computation, variable of the other computation and a final result variable) and the result of the 
computation. If the result is @False@, then it sets the final result. Otherwise, it waits until the 
other computation completes and then sets the final result.

\begin{comment}
\newpage
\section{Proofs}
\todo{This will not fit into the 12 page limit, but we can leave it online. Anyway,
the section will include proofs to demonstrate that the syntactic transformations
can be really done if the laws hold. This is not particularly interesting (technically)
--- just basic calculations using laws, but it is probably good to have it --- especially
because mostly did that already.}

% 1) Binding equivalence cancellation (MonadDuplicate)
% 2) Argument ordering   associativity & commutativity of mzip
% 3) Clause ordering     special law aboutt morelse
% 4) Alternative nointef monoid unit of morelse & right zero of binding
% 5) Argument nointef    monoid zero of mzip    & left zero of binding & unit of morelse
% 6) Matching units      zipping units of mzip
% 7) Matching images     naturaility   of mzip
% 8) Matching duplicate  zipping dupl. of mzip
% 9) Flattening          distributivity


The following is the proof of binding equivalence using the laws about \ident{MonadDuplicate}:
\begin{code}
  mduplicate m >>= \m -> (m >>= \v -> return e) >>= id
= mduplicate m >>= \m -> m >>= \v -> (\v -> return e) v >>= id  -- associativity
= mduplicate m >>= \m -> m >>= \v -> return e >>= id            -- eta conversion
= mduplicate m >>= \m -> m >>= \v -> id e                       -- left identity
= mduplicate m >>= \m -> id m >>= \v -> e                       -- add/remove id
= (mduplicate m >>= id) >>= (\v -> e)                           -- associativity back
= (join (mduplicate m)) >>= (\v -> e)                           -- definition of >>=
= m >>= (\v -> e)                                               -- mduplicate law
\end{code}

\begin{code}
mduplicate m  >>= \m ->
mduplicate n1 >>= \n1 ->
mduplicate n2 >>= \n2->
(m * n1 >>= f1) +
(m * n2 >>= f2)

mduplicate m  >>= \m ->
mduplicate ( mduplicate n1 >>= \n1 ->
             mduplicate n2 >>= \n2-> 
             (n1 >>= g1) + (n2 >>= g2) ) >>= f ->
(f * m) >>= f

mduplicate m  >>= \m ->
mduplicate ( mduplicate n1 >>= \n1 -> mduplicate n2 >>= \n2-> (n1 >>= g1) + (n2 >>= g2) ) >>= f ->
(f * m) >>= f

mduplicate m  >>= \m ->
mduplicate n1 >>= \n1 -> 
mduplicate n2 >>= \n2 -> 
(n1 >>= g1) + (n2 >>= g2) * m >>= ff
  
  

    (mduplicate m >>= f) * n ==  mduplicate m >>= \m -> f m * n
    mduplicate (mduplicate m >>= k) >>= l ==  mduplicate m >>= l . k

    mduplicate mzero >>= f == f mzero
    mduplicate (return a) >>= f == f (return a)
    mduplicate (map f m) >>= g == mduplicate m >>= g . map f


mduplicate a >>= \a ->
mduplicate a >>= \b ->
a * b >>= f


    (mduplicate m >>= f) * n
==  mduplicate m >>= \m -> f m * n

    mduplicate (mduplicate m >>= k) >>= l
                     mma      ma -> mb
                     mb               
                mmb                  mb -> mc
                     mc
==  mduplicate m >>= l . k
\end{code}
\end{comment}



\end{document}








