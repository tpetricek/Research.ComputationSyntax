\documentclass[runningheads,a4paper]{llncs}

\usepackage{amssymb}
\setcounter{tocdepth}{3}
\usepackage{graphicx}
\usepackage{multicol}
\usepackage{amsmath}
\usepackage{semantic}
\usepackage{verbatim}
\usepackage{stmaryrd}
\usepackage{color}
\usepackage{textcomp}
\usepackage{hyperref}
\usepackage{breakurl}

\begin{document}

\title{The F\# Computation Expressions Zoo}
\author{Tomas Petricek\inst{1} \and Don Syme\inst{2}}
\institute{University of Cambridge, UK\and Microsoft Research Cambridge, UK \\
\textsf{tp322@cam.ac.uk}, \textsf{dsyme@microsoft.com}}


\maketitle

% ==================================================================================================

\newcommand{\sep}[0]{\; | \;}
\newcommand{\kvd}[1]{\textnormal{\bfseries\sffamily #1}}
\newcommand{\plc}[1]{\textnormal{\emph{#1}}}
\newcommand{\ident}[1]{\textnormal{\sffamily #1}}

\newcommand{\cexpr}{\plc{cexpr}}
\newcommand{\expr}{\plc{expr}}
\newcommand{\pat}{\plc{pat}}

% ==================================================================================================

\begin{abstract}
Many computations can be structured using abstract types such as monoids, monad transformers or 
applicative functors. Functional programmers use those abstractions directly, but main-stream
languages often integrate concrete instances as language features -- e.g. generators in Python
or asynchronous computations in C\# 5.0. The question is, is there a sweet spot between convenient 
but inflexible language feature and flexible, but more difficult to use library?

\quad F\# \emph{computation expressions} answer this question in affirmative. Unlike 
the do notation in Haskell, computation expressions are not tied to a single kind of abstraction.
They support a wide range of computations, depending on what operations are available. They also 
provide greater syntactic flexibility leading to a more intuitive syntax.

\quad We show that computation expressions can structure well-known computations such as monoidal 
list comprehensions, monadic parsers, applicative formlets and asynchronous sequences based on the 
list monad transformer. We also present typing for computation expressions that is capable of 
capturing all these applications.
\end{abstract}

% ==================================================================================================

\section{Introduction}
Structures like monads [1] provide a way for composing computations with additional features. 
There are many examples -- monads can be composed using monad transformers [2], applicative
functors provide a more general abstraction useful for web programming [3] and additive monads
are useful for parsers [4].

In Haskell, we can write such computations using a mix of combinators and syntactic extensions like 
monad comprehensions [19] and do notation. On the other hand, languages such as Python and C\# 
emphasize the syntax and provide single-purpose support for asynchrony [20] and list generators [11]. 

We believe that syntax matters -- a language should provide \emph{uniform} syntactic support that 
can capture different abstractions, but is \emph{adaptable} and enables appropriate syntax depending
on the abstraction. This paper shows that F\# computation expressions provide such mechanism.

Although the technical aspects of the feature have been described before\footnote{F\# 3.0 extends the 
mechanism further to accomodate extensible query syntax. To keep this paper focused, we leave analysis 
of these extensions to future work.} [17], this paper is novel in that it relates the mechanism to 
well-known abstract computations. We also present new typing based on those uses.

\subsubsection{Practical examples.} 
We demonstrate the breath of computations that can be structured using F\# computation expressions.
The applications include asynchronous workflows and sequences \S\ref{sec:intro-async}, 
\S\ref{sec:intro-asyncseq}, list comprehensions \S\ref{sec:intro-seq}, monadic parsers \S\ref{sec:intro-parsers}
and formlets for web programming \S\ref{sec:intro-formlets}.

\vspace{-1em}
\subsubsection{Abstract computations.} We show that the above examples fit well-known types
of abstract computations, including additive monads and monad transformers, and we show what 
syntactic equalities hold as a result \S\ref{sec:laws}.

\vspace{-1em}
\subsubsection{Syntax and typing.} We revisit the definitions of computation expressions. We provide
typing rules that capture idiomatic uses \S\ref{sec:semantics-typing}, extend the translation
to support applicative functors \S\ref{sec:applicative} and discuss the threatment of effects
\S\ref{sec:delayed} that is needed in impure language.

\vspace{1em}
\noindent
We believe that software artifacts in programming language research matter [99], so all examples 
with implementations can be found and interactively run online: \url{http://tryjoinads.org/computations}. 
The syntax for applicative functors is a reserch extension; all other examples can be 
compiled with F\# 2.0.

% ==================================================================================================
\section{Computation expressions by example}
\label{sec:intro}

Computation expressions are blocks of code that represent computation with some non-standard 
aspect such as laziness, asynchronous evaluation, hidden state or other. The code inside the
block is re-interpreted using \emph{computation builder}. The computation builder is a record 
of operations that define the computation. It also defines what syntax is available in the block.

Computation expressions mirror the standard F\# syntax (let binding, loops, exception handling),
but support additonal computational constructs. For example \ident{let!} represents computational 
(monadic) alternative of let binding.

% --------------------------------------------------------------------------------------------------
\subsection{Monadic asynchronous workflows}
\label{sec:intro-async}

\begin{equation*}
\begin{array}{l}
\kvd{let}~\ident{getLength}~(url)=\ident{async}~\{\\
\quad \kvd{let!}~html = \ident{fetchAsync}(url)\\
\quad \kvd{do!}~\ident{Async}.\ident{Sleep}(1000)\\
\quad \kvd{return}~html.\ident{Length}\\
\}
\end{array}\qquad
\begin{array}{l}
\kvd{async}~\ident{Task}\langle\ident{string}\rangle~\ident{GetLength}(\kvd{string}~url)~\{\\
\quad \kvd{var}~html = \kvd{await}~\ident{FetchAsync}(url);\\
\quad \kvd{await}~\ident{Task}.\ident{Delay}(1000);\\
\quad \kvd{return}~html.\ident{Length};\\
\}
\end{array}
\end{equation*}

\begin{equation*}
\begin{array}{l}
\ident{async}.\ident{Bind}(\ident{fetchAsync}(url), \kvd{fun}~html\rightarrow\\
\qquad \ident{async}.\ident{Bind}(\ident{Async}.\ident{Sleep}(1000), \kvd{fun}~()\rightarrow\\
\qquad\qquad \ident{async}.\ident{Return}(html.\ident{Length})))
\end{array}
\end{equation*}


\subsubsection{Effects}

\begin{verbatim}
async { if delay then do! Async.Sleep(1000) 
        printfn "Starting..."
        return! asyncFetch url }
\end{verbatim}

a

\begin{verbatim}
async.Run(async.Delay(fun () -> 
  async.Combine(
    ( if delay then 
        async.Bind(Async.Sleep(1000), fun () -> async.Zero()) 
      else async.Zero() ), 
    async.Delay(fun () -> 
      printfn "Starting..."
      async.ReturnFrom(asyncFetch url)) )))
\end{verbatim}
imperative monads with sequencing and effects

\subsection{Additive sequence expressions}
\label{sec:intro-seq}

\subsection{Additivie parser combinators}
\label{sec:intro-parsers}

\subsection{Layered asynchronous sequences}
\label{sec:intro-asyncseq}

\subsection{Applicative formlets}
\label{sec:intro-formlets}

% ==================================================================================================
\section{Semantics of computation expressions}
\label{sec:semantics}

Computation expressions are blocks representing non-standard computations – that is, computation 
that have some additional aspect, such as laziness, asynchronous evaluation, hidden state or other. 
The code inside the block mirrors the standard F\# syntax, but it is re-interpreted in the context of 
a non-standard computation. Computation expressions may also include a number of constructs that 
provide non-standard alternatives of standard constructs. For example, the let! syntax provides 
non-standard (monadic) version of let binding. 

In this section, we use two examples to show how computation expressions unify single-purpose extensions 
from other languages. Then we look at the formal definition in the F\# specification [17].


\begin{equation*}
\begin{array}{lcl}
\kvd{def}~\ident{duplicate}(list): &\hspace{5em}&\ident{duplicate}~list =\\
\quad \kvd{for}~n~\kvd{in}~list:               &&\quad \kvd{do}~n~\leftarrow~list\\
\quad\quad \kvd{yield}~n                       &&\quad \quad~\, (\ident{return}~n)~`\ident{mplus}`\\
\quad\quad \kvd{yield}~n*10                    &&\quad \quad\quad~\, (\ident{return} \$)\\
\end{array}
\end{equation*}
\begin{verbatim}
duplicate list = 
  do a <- list
     (return a) `mplus` (return $ a * 10)
\end{verbatim}


\newpage

% --------------------------------------------------------------------------------------------------

\subsection{Syntax}
\label{sec:semantics-syntax}

\begin{equation*}
\begin{array}{lclcl}
\expr  &=& \expr ~\{~ \cexpr ~\}                      &\quad&\textnormal{(computation expression)}\\
\cexpr &=& \kvd{let}~v=\expr~\kvd{in}~\cexpr                &&\textnormal{(binding value)} \\
       &|& \kvd{let!}~v=\expr~\kvd{in}~\cexpr               &&\textnormal{(binding computation)} \\
       &\begin{array}{c}|\vspace{1em}\end{array}&\hspace{-0.15em}
           \begin{array}{l}
             \kvd{let!}~v_1=\expr_1~\kvd{and}~\ldots\\[-0.2em]
             \kvd{and}~v_n=\expr_n ~\kvd{in}~\cexpr 
           \end{array}                                      &&\textnormal{(parallel computation binding)} \\
       &|& \kvd{for}~v~\kvd{in}~\expr~\kvd{do}~\cexpr       &&\textnormal{(for loop computation)} \\
       &|& \kvd{return}~\expr                               &&\textnormal{(return value)} \\
       &|& \kvd{return!}~\expr                              &&\textnormal{(return computation)} \\
       &|& \kvd{yield}~\expr                                &&\textnormal{(yield value)} \\
       &|& \kvd{yield!}~\expr                               &&\textnormal{(yield computation)} \\
       &|& \cexpr_1; \cexpr_2                               &&\textnormal{(compose computations)} \\       
       &|& \expr                                            &&\textnormal{(effectful expression)} \\
\end{array}
\end{equation*}

% --------------------------------------------------------------------------------------------------

\subsection{Typing}
\label{sec:semantics-typing}

Typing of yield is similar 

\newcommand{\mtyp}[1]{M #1}
\newcommand{\ntyp}[1]{N #1}
\newcommand{\ltyp}[1]{L #1}
\newcommand{\dtyp}[1]{D #1}

\newcommand{\tya}[2]{#1\hspace{-0.15em}:\hspace{-0.15em}#2}
\newcommand{\cvdash}{\vartriangleright_\sigma}
\newcommand{\unit}{\ident{unit}}

\begin{equation*}
\begin{array}{rl}
\inference 
  { \Gamma \vdash \expr : \sigma  & \Gamma \cvdash \cexpr : \mtyp{\tau} }
  { \Gamma \vdash \expr ~\{~ \cexpr ~\} : \ntyp \tau }
&\quad(\sigma.\plc{run}~:~\mtyp{\tau} \rightarrow \ntyp \tau )  
\\[1.5em]
\inference
  { \Gamma \vdash \expr : \tau_1 &
    \Gamma, \tya{v}{\tau_1} \cvdash \cexpr : \mtyp{\tau_2} }
  { \Gamma \cvdash \kvd{let}~v=\expr~\kvd{in}~\cexpr : \mtyp{\tau_2}  }
\\[1.5em]
\inference
  { \Gamma \vdash \expr : \mtyp{\tau_1} &
    \Gamma, \tya{v}{\tau_1} \cvdash \cexpr : \ntyp{\tau_2} }
  { \Gamma \cvdash \kvd{let!}~v=\expr~\kvd{in}~\cexpr : \ntyp{\tau_2}  }
&\quad(\sigma.\plc{bind}~:~\mtyp{\alpha} \rightarrow (\alpha \rightarrow \ntyp{\beta}) \rightarrow \ntyp{\beta} )  
\\[1.5em]
\inference
  { \Gamma \vdash \expr : \mtyp{\tau_1} &
    \Gamma, \tya{v}{\tau_1} \cvdash \cexpr : \ntyp{\tau_2} }
  { \Gamma \cvdash \kvd{for}~v~\kvd{in}~\expr~\kvd{do}~\cexpr : \ntyp{\tau_2}  }
&\quad(\sigma.\plc{for}~:~\mtyp{\alpha} \rightarrow (\alpha \rightarrow \ntyp{\beta}) \rightarrow \ntyp{\beta} )  
\\[1.5em]
\inference
  { \Gamma \vdash \expr : \tau }
  { \Gamma \cvdash \kvd{return}~\expr : \mtyp{\tau}  }
&\quad(\sigma.\plc{return}~:~\alpha \rightarrow \mtyp{\alpha})
\\[1.5em]
\inference
  { \Gamma \vdash \expr : \mtyp \tau }
  { \Gamma \cvdash \kvd{return!}~\expr : \ntyp{\tau}  }
&\quad(\sigma.\plc{returnFrom}~:~\mtyp{\alpha} \rightarrow \ntyp{\alpha})  
\\[1.5em]
\inference
  { \Gamma \cvdash \cexpr_1 : \mtyp{\tau_1}  &  \Gamma \cvdash \cexpr_2 : \ntyp{\tau_2}}
  { \Gamma \cvdash \cexpr_1; \cexpr_2 : \ltyp{\tau_1}  }
&\quad\left(\begin{array}{lll}
\plc{delay}   &:& (\unit \rightarrow \ntyp{\alpha}) \rightarrow \dtyp{\alpha} \\
\plc{combine} &:& \mtyp{\tau_1} \rightarrow \dtyp{\tau_2} \rightarrow \ltyp{\tau_2}
\end{array}\right)
\\[1.5em]
\inference
  { \Gamma \vdash \expr : \unit }
  { \Gamma \cvdash \expr : \mtyp{\tau}  }
&\quad(\sigma.\plc{zero}~:~\unit \rightarrow \mtyp{\tau})
\end{array}
\end{equation*}

When we write $\alpha$ and $\beta$, we assume universal quantification. When we write $\tau$, we 
mean any instantiation (but the operation may not be universally qualified).

For example, \plc{zero} may have a type \mtyp{\unit} or $\mtyp{\alpha}$.

Typing of \ident{yield} and \ident{yield!} is similar to the typing of \ident{return} and \ident{return!},
so we omit them.

Zero may 

% --------------------------------------------------------------------------------------------------

\subsection{Translation}
\label{sec:semantics-transl}

% ==================================================================================================

\section{Delayed computations}
\label{sec:delayed}

% ==================================================================================================

\section{Abstract computation types}
\label{sec:laws}

% ==================================================================================================

\section{Applicative computations}
\label{sec:applicative}

% ==================================================================================================

\section{Conclusions}
\label{sec:conclusions}

% ==================================================================================================

\section*{Acknowledgements}



\bibliographystyle{abbrv}
\bibliography{computation-zoo}

\appendix

\section{Bonus}




\end{document}












