\documentclass{tmr}
\usepackage{mflogo}

%\usepackage{ulem}
%\usepackage{verbatim}

%include lhs2TeX.fmt
%include lhs2TeX.sty
%include polycode.fmt

%\newcommand{\authornote}[3]{{\color{#2} {\sc #1}: #3}}
%\newcommand\dom[1]{\authornote{dominic}{blue}{#1}}
%\newcommand\dout[1]{ {\color{blue}{\sout{#1}}} }
%\newcommand\din[1]{ {\color{blue}{{#1}}} }

\title{Fun with Parallel Monad Comprehensions}
\author{Tomas Petricek\email{tomas.petricek@cl.cam.ac.uk}}

\begin{document}
\newcommand{\fsharp}[0]{F$^\#$}
\newcommand{\ident}[1]{{\normalfont\sffamily #1}}

% ==================================================================================================

\begin{introduction} 
Monad comprehensions have an interesting history. They were the first syntactic extension for 
programming with monads. They were implemented in Haskell, but later replaced with plain list
comprehensions and monadic do notation. Now, monad comprehensions are back in Haskell,
more powerful than ever before!

Redesigned monad comprehensions generalize the syntax for working with lists. Quite interestingly, 
they also generalize syntax for zipping, grouping and ordering of lists. This article, shows how to 
use some of the new expressive power when working with well-known monads. You'll learn what 
``parallel composition'' means for parsers, poor man's concurrency monad and an evaluation 
order monad.
\end{introduction}

% ==================================================================================================

\section{Introduction}

This article is inspired by my earlier work on \textit{joinads} \cite{joinads}, an extension that 
adds pattern matching on abstract values to the \textit{computation expression} syntax in \fsharp. 
Computation expressions are quite similar to the do notation in Haskell, but there are some notable 
differences. I'll briefly write about them later. After implementing the \fsharp \ version of 
joinads, I wanted to see how the ideas would look in Haskell. I was quite surprised to find out that 
a recent patch for GHC adds some of the expressive power of joinads to Haskell. 

To add some background -- the \fsharp \ computation expression syntax can be used to work with 
\textit{monads}, but also with \textit{monoids} and a few other abstract notions of computation. 
It also adds several constructs that generalize imperative features of \fsharp \ including |while| 
and |for| loops as well as exception handling. The \textit{joinads} extension adds support for
pattern-matching on ``monadic values''. For example, you can define a parallel programming monad
and use joinads to wait until two parallel computations both complete or wait until the first of 
the two completes returning a value matching a particular pattern.

How are \fsharp \ joinads related to Haskell? A recent GHC patch implemented by Nils Schweinsberg 
\cite{bringbackmc, comprefun} brings back the support for monad comprehensions to Haskell. 
The patch doesn't just re-implement original monad comprehensions, but it also generalizes recent 
additions to list comprehensions allowing parallel monad comprehensions and monadic 
versions of operations like ordering and grouping \cite{groupordercompre}. 

The operation that generalizes parallel comprehensions is closely related to a |merge| operation 
that I designed for \fsharp \ joinads. In the rest of this article, I demonstrate some of the 
interesting programs that can be written using this operation and the elegant syntax provided by 
the re-designed monad comprehensions. 

% --------------------------------------------------------------------------------------------------

\subsection{Quick review of list comprehensions}
List comprehensions are a very powerful mechanism for working with lists in Haskell. I expect that 
you're already familiar with them, but let me start with a few examples. I will use the examples 
later to demonstrate how the generalized monad comprehension syntax works in a few interesting cases.

If we have a list |animals| containing ``cat'' and ``dog'' and a list |sounds| containing 
animal sounds ``meow'' and ``woof'', we can write the following snippets:

\begin{verbatim}
> [ a ++ " " ++ s | a <- animals, s <- sounds ]
["cat meow","cat woof","dog meow","dog woof"]

> [ a ++ " " ++ s | a <- animals, s <- sounds, a !! 1 == s !! 1 ]
["dog woof"]

> [ a ++ " " ++ s | a <- animals | s <- sounds ]
["cat meow","dog woof"]
\end{verbatim}
The first example uses just the basic list comprehension syntax. It uses two \textit{generators}
to implement a join of the two collections. The second example adds a guard to specify that we
want only pairs of strings whose second character is the same. The guard serves as an additional
filtering of the results.

The last example uses parallel list comprehensions. The syntax is available after enabling the 
\ident{ParallelListComp} language extension. It allows us to take elements from multiple lists, so
that a n$^\text{th}$ element of the first list is matched with the n$^\text{th}$ element of the 
second list. The same functionality can be easily expressed using the |zip| function. 

% --------------------------------------------------------------------------------------------------

\subsection{Generalizing to monad comprehensions}
The three examples we've seen in the previous section are straightforward when working with lists.
When you apply the patch that implements monad comprehensions \cite{bringbackmc} and turn on the
\ident{MonadComprehensions} language extension, you can use the same syntax for working
with further notions of computations. If we implement appropriate type classes, we can even use 
guards, parallel comprehensions and operations like ordering or grouping. 
Figure~\ref{fig:monad-defs} shows the type classes and functions that are used by the desugaring.

\begin{figure}
\begin{verbatim}
class Monad m where
  (>>=)  :: m a -> (a -> m b) -> m b
  return :: a -> m a

class (Monad m) => MonadPlus m where
  mzero :: m a
  mplus :: m a -> m a -> m a

class (Monad m) => MonadZip m where
  mzip :: m a -> m b -> m (a, b)

guard :: MonadPlus m => Bool -> m ()
guard b = if b then return () else mzero
\end{verbatim}
\caption{Type-classes and functions used by monad comprehensions}
\label{fig:monad-defs}
\end{figure}

Aside from \ident{Monad}, the desugaring also uses \ident{MonadPlus} and \ident{MonadZip} 
type-classes. The former is used only for the |guard| function, which is also defined in the 
figure. The latter is a new class that has been added as a generalization of parallel
list comprehensions. The name of the function makes it clear that the type-class is a generalization
of the |zip| function. The patch also defines a \ident{MonadGroup} type-class that generalizes 
grouping operations inside list comprehensions, but I will not use the feature in this article, 
so I omitted the class from the figure.

You can find the general desugaring rules in the patch description \cite{bringbackmc}. In this 
article, we'll just go through the examples from the previous section and examine how the 
translation looks. The following declaration shows how to implement the \ident{Monad} type class
for lists:

\begin{verbatim}
instance Monad [] where
  source >>= f = concat $ map f source
  return a = [a]
\end{verbatim}
The |>>=| operation, called bind, applies the provided function to each element of the input list 
and then concatenates the generated lists. The |return| function creates a singleton list containing 
the specified value. The |mzero| value from \ident{MonadPlus} type-class is an empty list, which 
means that |guard| returns |[()]| when the argument is |True| and empty list otherwise. Finally,
the |mzip| function for lists is just |zip|.

Now we have everything we need to look at the desugaring of monad comprehensions. The first
example from the previous section used multiple generators and can be translated purely in terms
of \ident{Monad}:

\begin{verbatim}
animals >>= (\a -> sounds >>= (\s -> 
  return $ a ++ " " ++ b))
\end{verbatim}
Every \textit{generator} is translated to a binding using |>>=|. The operations are nested and
the innermost operation always returns the result of the output function. The next snippet shows
what happens when we add predicate to filter the results:

\begin{verbatim}
animals >>= (\a -> sounds >>= (\s -> 
  guard (a !! 1 == s !! 1) >>= (\_ -> 
    return $ a ++ " " ++ s) ))
\end{verbatim}
A predicate is translated into a call to the |guard| function in the innermost part of the 
desugared expression. When the function returns |mzero| value (an empty list), the result of the
binding will also be |mzero|, so the element for which the predicate doesn't hold will be 
filtered out. Finally, let's look at the translation of the last example:

\begin{verbatim}
(animals `mzip` sounds) >>= (\(a, s) -> 
  return $ a ++ " " ++ s)
\end{verbatim}
When we use parallel comprehensions, the inputs of the generators are combined using the |mzip|
function. The result is passed to the bind operation, which applies the output function to values
of the combined computation. If we also specified filtering, the |guard| function would be added 
to the innermost expression as in the previous example.

As you can see, the translation of monad comprehensions is quite simple, but it adds an additional
expressivity to the syntax for working with monads. In particular, the do notation doesn't provide
an equivalent syntactic extension for writing parallel comprehensions. Constructs like generalized 
ordering (using functions of type |m a -> m a|) and generalized grouping (using functions 
|m a -> m (m a)|) add even more expressivity, but that would be a topic for another article. 
In the next three sections, I show how we could implement the |mzip| operation for several 
interesting monads, starting with monadic parsers. At the end of the article, I also briefly 
consider laws that should hold about the |mzip| operation.

% ==================================================================================================

\section{Composing parsers in parallel}
What does a \textit{parallel composition of two parsers} mean? Probably the best thing we can do 
is to run both parsers on the input string and return a tuple with the two results. That sounds
quite simple, but what is this construct good for? Let's first implement it and then look at 
some examples. 

% --------------------------------------------------------------------------------------------------

\subsection{Introducing parsers}
Many of you may already know monadic parsing combinators. I include all definitions that are 
necessary to build a complete example for two reasons -- to make this article self-contained
and because there is one minor, but important twist.

A parser is a function that takes the 
input string and returns a list of possible results. It may be empty (if the parser fails) or 
contain several items (if there are multiple ways to parse the input). The implementation I use 
in this article mostly follows the one by Hutton and Meijer \cite{monadparsing}:

\begin{verbatim}
newtype Parser a
  = Parser (String -> [(a, Int, String)])
\end{verbatim}
The result of parsing is a tuple containing a value of type |a| produced by the parser, a number
of characters consumed by the parser and the remaining unparsed part of the string. The |Int| value
is not usually included in parser definitions. We'll need it later in the implementation of |mzip|.

Now that we have a definition of parsers, we can create our first primitive parser and a 
function that runs parser on an input string and returns the results:

\begin{verbatim}
item :: Parser Char
item = Parser (\input -> case input of
  "" -> []
  c:cs -> [(c, 1, cs)])
  
run :: Parser a -> [a]
run (Parser p) input = 
  [ result | (result, _, tail) <- p input, tail == [] ]
\end{verbatim}
The |item| parser returns the first character of the input string. When it succeeds, it consumes 
a single character, so it returns 1 as the second element of the tuple. The |run| function
applies the underlying function of the parser to a specified input. As specified by the condition
|tail == []|, the function returns results of those parsers which parsed the entire input.
The next step is to make the parser monadic.

% --------------------------------------------------------------------------------------------------

\subsection{Implementing the parser monad}
Parsers are a well known example of \textit{monads} and \textit{monoids}. This means that we can 
implement both \ident{Monad} and \ident{MonadPlus} type classes. You
can find the implementation in Figure~\ref{fig:monad-instance}.

\begin{figure}
\begin{verbatim}
instance Monad Parser where
  return a = Parser (\input -> [(a, 0, input)])
  (Parser p1) >>= f = Parser (\input ->
    [ (result, n1 + n2, tail) 
        | (a, n1, input') <- p1 input
        , let (Parser p2) = f a
        , (result, n2, tail) <- p2 input' ])

instance MonadPlus Parser where 
  mzero = Parser (\_ -> [])
  mplus (Parser p1) (Parser p2) = Parser (\input ->
    p1 input ++ p2 input)
\end{verbatim}
\caption{Instances of \ident{Monad} and \ident{MonadPlus} for parsers}
\label{fig:monad-instance}
\end{figure}

The |return| operation returns a single result containing the specified value that doesn't consume
any input. The |>>=| operation can be implemented using ordinary list comprehensions.
It runs the parsers in sequence, returns the result of the second parser and consumes the sum of 
characters consumed by the first and the second parser. The |mzero| operation creates a parser that 
always fail and |mplus| represents a non-deterministic
choice between two parsers. 

The two type class instances allow us to use some of the monad comprehension syntax. We can now 
use the |item| primitive to write a few simple parsers:

\begin{verbatim}
sat pred = [ ch | ch <- item, pred ch ]
char ch = sat (ch ==)
notChar ch = sat (ch /=)

some p = [ a:as | a <- p, as <- many p ]
many p = some p `mplus` return []
\end{verbatim}

The |sat| function creates a parser that parses a character matching a specified predicate. 
The \textit{generator syntax} |ch <- item| corresponds to monadic binding and is desugared 
into an application of the |>>=| operation. Because the \ident{Parser} type is an instance
of \ident{MonadPlus}, we can use a guard to apply a predicate such as |pred ch|. The 
desugared version of the function is:

\begin{verbatim}
sat pred = item >>= (\ch ->
  guard (pred ch) >>= (\_ -> return ch))
\end{verbatim}

The |some| and |many| combinators are mutually recursive. The first one creates a parser that 
parses one or more occurrences of |p|. We encode it using monad comprehension with two bindings. 
The parser parses |p| followed by |many p|. The same thing could be written using do notation, but 
monad comprehensions provide quite elegant and succinct alternative, with a syntactic support for 
guards (as you can see in the first example).

The order of monadic bindings usually matters. The monad comprehension syntax makes this fact
perhaps slightly less obvious than the do notation. To demonstrate this, let's look at a parser that
parses the body of an expression enclosed in brackets:

\begin{verbatim}
brackets :: Char -> Char -> Parser a -> Parser a
brackets op cl body = 
  [ inner 
      | _ <- char op
      , inner <- brackets op cl `mplus` body
      , _ <- char cl ]

skipBrackets = brackets '(' ')' (many item)
\end{verbatim}
The combinator takes characters representing opening and closing brackets and a parser for parsing 
the body inside brackets. It uses a monad comprehension with three binding expressions that parse
opening brace, the body or more brackets and then the closing brace.

If you run the parser using |run skipBrackets "((42))"| you get a list containing |"42"|, but also 
|"(42)"|. This is because the |many item| parser can also consume brackets. To correct that, we need
to write a parser that accepts any character except opening and closing brace. As you'll see 
shortly, this can be elegantly solved using parallel comprehensions.

% --------------------------------------------------------------------------------------------------

\subsection{Parallel composition of parsers}
To support parallel monad comprehensions, we need to implement \ident{MonadZip}. 
As a reminder, the type class defines an operation |mzip| with the following type:

\begin{verbatim}
mzip :: m a -> m b -> m (a, b)
\end{verbatim}
By looking just at the type signature, you can see that the operation can be implemented in terms
of |>>=| and |return|. This is a reasonable definition for some monads, such as the |Reader| monad,
but not for all of them. For example, |mzip| for lists uses a different operation. 
The implementation for parsers in Figure~\ref{fig:monadzip-instance} also isn't expressed using
other monad primitives.

\begin{figure}
\begin{verbatim}
instance MonadZip Parser where 
  mzip (Parser p1) (Parser p2) = Parser (\input -> 
    [ ((a, b), n1, tail1) 
        | (a, n1, tail1) <- p1 input
        , (b, n2, tail2) <- p2 input
        , n1 == n2 ])
\end{verbatim}
\caption{Instance of \ident{MonadZip} type class for parsers}
\label{fig:monadzip-instance}
\end{figure}

The parser created by |mzip| independently parses the input string using both of the parsers.
It uses list comprehensions to find all combinations of results such that the number of 
characters consumed by the two parsers was the same. For each matching combination, the parser 
returns a tuple with the two parsing results. Requiring that the two parsers consume the same 
number of characters is not an arbitrary decision. It means that the remaining unconsumed strings 
|tail1| and |tail2| are the same and so we can return any of them. I use a counter instead of 
comparing strings, because it allows us to work with possibly infinite
strings. 

Let's get back to the example with parsing brackets. The following snippet uses parallel 
monad comprehensions to create a version that consumes all brackets:

\begin{verbatim}
skipAllBrackets = brackets '(' ')' body
  where body = many [ c | c <- notChar '(' | _ <- notChar ')' ]
\end{verbatim}
The parser |body| takes zero or more of any characters that are not opening or closing brackets.
The parallel comprehension runs two |notChar| parsers on the same input. They both read a single
character and they succeed if the character is not `(' and `)' respectively. The resulting parser
succeeds only if both them succeed. Both of the parallel parsers return the same character, so we
assign the first one to a symbol |c| and return it as the result and ignore the second one.

Another example where this syntax may be useful is validation of inputs. For example, a Cambridge
phone number consists of 10 symbols, needs to contain only digits and starts with 1223. The new
syntax allows us to directly encode these three rules:

\begin{verbatim}
cambridgePhone = 
  [ n | n <- many (sat isDigit)
      | _ <- times item 9 
      | _ <- startsWith (string "1223") ]
\end{verbatim}
The encoding is quite straightforward. We need some additional combinators such as |times| that
repeats a parser specified number of times and |startsWith| that runs a parser and then consumes
any number of characters. 

We could construct a single parser that recognizes valid Cambridge phone number without using
|mzip|. The point of this example is that we can quite nicely combine several independent rules,
which makes the validation code easy to understand and extend.

% --------------------------------------------------------------------------------------------------

\subsection{Parallel composition of context-free parsers}

Monadic parser combinators are very expressive. In fact, they are often \textit{too expressive}, 
which makes it difficult to implement the combinators efficiently. This was a motivation for
the development of non-monadic parsers, such as the one by Swierstra \cite{deterparser}, 
\cite{parsingtutorial}. They are less expressive, but can be more efficient.  
A weaker interface that can be used for writing parsers is \textit{applicative functor} developed 
by McBride and Paterson \cite{applicative}. You can find a Haskell type class \ident{Applicative}
that represents applicative functors in Figure~\ref{fig:applicative}.
\begin{figure}
\begin{verbatim}
class (Functor f) => Applicative f where
  pure  :: a -> f a
  (<*>) :: f (a -> b) -> f a -> f b
\end{verbatim}
\caption{Definition of the \ident{Applicative} type class}
\label{fig:applicative}
\end{figure}

There is also an alternative definition of \ident{Applicative} which uses an operation with 
the same type signature as our |mzip|. As I explain in the next section, our |mzip| is different
than the one we could define in terms of |pure| and |<*>|. In some sense, we will be combining
two different applicative functors in this section. However, we'll get back to this topic later.

The usual applicative parsers allow us to write parsers where the choice of the next parser 
doesn't depend on the value parsed so far. In terms of formal language theory, they can express 
only \textit{context-free} languages. This is still sufficient for many practical purposes. For 
example our earlier |brackets| parser can be written using the applicative combinators:

\begin{verbatim}
brackets op cl body = 
  pure (\_ inner _ -> inner)
    <*> char op
    <*> brackets op cl body `mplus` body
    <*> char cl
\end{verbatim}
The example creates a parser that always succeeds and returns a function using the |pure| 
combinator. Then it applies this function (contained in a parser) to three arguments (produced
by the three parsers). You don't need to understand the example in details. The important
point is -- if we have a comprehension where the next \textit{generator} in a sequence
doesn't depend on the value produced by the previous one, we can rewrite it using the 
\ident{Applicative} interface. For parsers, this means that we're parsing just a 
\textit{context-free grammar}.

The interesting question is, what operation does |mzip|, as we defined it in the previous section,
represent for context-free grammars? A language we obtain if parses for two other languages both 
succeed is an \textit{intersection} of the two languages. An intersection of two context-free 
languages is not context-free, which can be demonstrated using the following example:

\[ 
A = \{ a^m b^m c^n \arrowvert m, n \geq 0 \} 
\qquad
B = \{ a^n b^m c^m \arrowvert m, n \geq 0 \} 
\]
\[ 
A \cap B = \{ a^m b^m c^m \arrowvert m \geq 0 \}
\]

The language $A$ accepts words that start with some number of `a' followed by the same
number of `b' and then arbitrary number of `c' symbols. The language $B$ is similar, but it
starts with a group of any length followed by two groups of the same lengths. Both of the
languages are context-free. In fact, our parser |brackets| can be used to parse the two 
character groups with same lengths.

The intersection $A \cap B$ is not context-free \cite{cflintersect}, but we can easily
encode it using parallel composition of parsers. We don't need the full power of monads
to parse the first group and calculate its length. It could be implemented just in terms of
\ident{Applicative} and |mzip|, but we use the nicer monad comprehension syntax:

\begin{verbatim}
[ True | _ <- many $ char 'a', _ <- brackets 'b' 'c' unit
       | _ <- brackets 'a' 'b' unit, _ <- many $ char 'c' ]
  where unit = return ()
\end{verbatim}
The example uses both parallel and sequential binding, but the sequential composition doesn't
contain dependencies on previously parsed results. It uses |brackets| to parse two groups of the
same length, followed (or preceded) by |many| to consume the remaining group of arbitrary length.

% --------------------------------------------------------------------------------------------------

\subsection{Applicative and parallel parsers}
Before moving to the next example, let's take a look how |mzip| relates to applicative functors.
An alternative definition of applicative functor (\cite{applicative} section 7) uses an operation
$\star$ with exactly the same type signature as |mzip|. Since applicative functors generalize
monads, we can always define $\star$ in terms of |>>=| and |return|. 

\UndefineShortVerb{\|}
\DefineShortVerb{\_}

Judging by the type, we could use $\star$ as our _mzip_ operation, but this definition may not
always be intuitively right. Referring to the intuition is always tricky, so I'll express my 
expectations more formally in terms of laws at the end of the article. However, if we always just 
used $\star$ as the _mzip_ operation, we wouldn't get any additional expressive power, so there 
would be no point in using parallel monad comprehensions. The expression _[ e | a <- e1 | b <- e2]_
would mean exactly the same thing as _[ e | a <- e1, b <- e2]_.

\UndefineShortVerb{\_}
\DefineShortVerb{\|}

This definitely isn't the case for lists. The definition of $\star$ for the usual \ident{List}
monad would give us a Cartesian product of lists. This isn't very useful, because we can get that
behavior using multiple generators. Instead, list comprehensions use zipping of lists, which comes 
from a different applicative functor, namely \ident{ZipList}.

The example with parsers is similar. The implementation of $\star$ would give us a sequential 
composition of parsers. This wouldn't be very useful, so I defined |mzip| as the intersection of 
parsers. This is an interesting operation that adds expressive power to the language. The |mzip| 
operation also defines an instance of applicative functor for parsers, but a different one.

For the fans of category theory, the $\star$ operation is a \textit{monoidal functor} defined by
the monad we're working with. The |mzip| operation is also a \textit{monoidal functor}, but it may 
be a different one. One of the additional laws that we can require about |mzip| is commutativity 
(more about that later), which means that |mzip| should be a \textit{symmetric monoidal functor}.

% ==================================================================================================

\section{Parallelizing cooperative computations}

As the name \textit{parallel} monad comprehensions suggests, we can use the syntax for 
running computations in parallel. Unlike comprehensions with multiple generators, parallel
comprehensions cannot have any dependencies between the composed bindings. This means that the 
computations can be evaluated independently. 

In this section, I demonstrate the idea using a poor man's concurrency monad inspired by Claessen's
work \cite{poorman}. The monad can be used to implement a lightweight cooperative concurrency. When 
running two computations in parallel, we can allow interleaving of atomic actions from the two 
threads.

% --------------------------------------------------------------------------------------------------

\subsection{Modelling resumable computations}

The example I demonstrate here models computations using \textit{resumptions}. This concept is 
slightly simpler than the original poor man's concurrency monad (which is based on continuations). 
A resumption is a computation that has either finished and produced some value or has not
finished, in which case it can run one atomic step and produce a new resumption:

\begin{verbatim}
data Monad m => Resumption m r 
  = Step (m (Resumption m r))
  | Done r
\end{verbatim}
The type is parametrized over a monad and a return type. When evaluating a resumption, we 
repeatedly run the computation step-by-step. While evaluating, we perform the effects allowed by
the monad |m| until we eventually get a result of type |r|. If you're interested in more details 
about the |Resumption| type, you can find a similar definition in Harrison's cheap threads 
\cite{cheapthreads} and Papaspyrou's resumption transformer \cite{resmonad}.

Now that we have a type, we can write a function to create a |Resumption| that runs a single
atomic action and then completes and a function that runs a |Resumption|:

\begin{verbatim}
run :: Monad m => Resumption m r -> m r
run (Done r) = return r
run (Step m) = m >>= run

action :: Monad m => m r -> Resumption m r
action a = Step [ Done r | r <- a ]
\end{verbatim}
The |run| function takes a resumption that may perform effects specified by the monad |m| and it
runs the resumption inside the monad until it reaches |Done|. The function runs the whole 
computation sequentially (and it cannot be implemented differently). The cooperative concurrency
can be added later by creating a combinator that interleaves the steps of two |Resumption|
computations.

The |action| function is quite simple. It returns a |Step| that runs the specified action 
inside the monad |m| and then wraps the result using the |Done| constructor. I implemented the
function using monad comprehensions, but we could also use the usual do notation, which may be
a better fit for effectful computations:

\begin{verbatim}
action' a = Step $ do r <- a; return (Done r)
\end{verbatim}
The function does exactly the same thing as the previous version. The comprehension syntax was 
slightly more succinct, because the computation ends with |return|, which is the standard structure
of comprehensions.

% --------------------------------------------------------------------------------------------------

\subsection{Implementing the resumption monad}
You can see the implementation of the \ident{Monad} type class in Figure~\ref{fig:poorman-instance}. 
The |return| operation creates a new resumption in the ``done'' state and contains the specified 
value. The |>>=| operation constructs a resumption that gets the result of the 
first resumption and then calls the function |f|. When the left parameter is |Done|, we 
apply the function to the result and wrap the application inside |return| (because the function
is pure) and |Step|. When the left parameter is |Step|, we create a resumption that runs the
step and then uses |>>=| recursively to continue running steps from the left argument until it 
finishes.

\begin{figure}
\begin{verbatim}
instance Monad m => Monad (Resumption m) where
  return a = Done a
  (Done r) >>= f = Step $ return (f r)
  (Step s) >>= f = Step $ do
    next <- s
    return $ next >>= f

instance MonadTrans Resumption where
  lift = action
\end{verbatim}
\caption{Instances of \ident{Monad} and \ident{MonadTrans} 
  for resumptions.}
\label{fig:poorman-instance}
\end{figure}

The listing also defines an instance of \ident{MonadTrans} to make |Resumption| a monad transformer.
The |lift| function takes a computation in the monad |m| and turns it into a computation in the 
|Resumption| monad. This is exactly what our function for wrapping atomic actions does.

Equipped with the two type class instances and the |run| function, we can write some interesting
computations. The following function creates a computation that runs for the specified number of
steps, prints some string in each step and then returns a specified value:

\begin{verbatim}
printLoop :: String -> Int -> a -> Resumption IO a
printLoop str count result = do
  lift $ putStrLn str
  if count == 1 then return result
  else printLoop str (count - 1) result

cats = run $ printLoop "meow" 3 "cat"
\end{verbatim}
The function is written using the do notation. It first prints the specified string, using 
|lift| to turn the |IO ()| action into a single-step |Resumption IO ()|. When the counter reaches
one, it returns the result; otherwise it continues looping.

The snippet defines a simple computation |cats| that prints ``meow'' three times and then 
returns a string ``cat''. The computations created by |printLoop| are not fully opaque. If we have 
two computations like |cats|, we can treat them as sequences of steps and interleave them. 
This is what the |mzip| operation does.

% --------------------------------------------------------------------------------------------------

\subsection{Parallel composition of resumptions}

If we have two resumptions, we can compose them to run in sequence either using the do notation or
using a monad comprehension with two generators. To run them in parallel, we need to implement
interleaving of the steps as shown in Figure~\ref{fig:poorman-zip}. 

\begin{figure}
\begin{verbatim}
instance Monad m => MonadZip (Resumption m) where
  mzip (Done a) (Done b) = Done (a, b)
  mzip sa sb = Step [ mzip a b | a <- step sa, b <- step sb ]
    where step (Done r) = return $ Done r 
          step (Step sa) = sa
\end{verbatim}
\caption{Instance of \ident{MonadZip} that composes resumptions in parallel}
\label{fig:poorman-zip}
\end{figure}

The result of |mzip| is a resumption that consists of multiple steps. In each step, it performs
one step of both of the resumptions given as arguments. When it reaches a state when both of the
resumptions complete and produce results, it returns a tuple containing the results using |Done|.
A step is performed using an effectful |step| function. To keep the implementation simple, we keep
applying |step| to both of the resumptions and then recursively combine the results. Applying
|step| to a resumption that has already completed isn't a mistake. This operation doesn't do 
anything and just returns the original resumption (without performing any effects).

Once we define |mzip|, we can start using the parallel comprehension syntax for working with 
resumptions. The next snippet demonstrates two ways of composing resumptions. In both examples, we 
compose a computation that prints ``meow'' two times and then returns ``cat'' with a computation 
that prints ``woof'' three times and then returns ``dog'':

\begin{verbatim}
animalsSeq = 
  [ c ++ " and " ++ d
     | c <- printLoop "meow" 2 "cat" 
     , d <- printLoop "woof" 3 "dog" ]

animalsPar = 
  [ c ++ " and " ++ d
     | c <- printLoop "meow" 2 "cat" 
     | d <- printLoop "woof" 3 "dog" ]
\end{verbatim}

The only difference between the two examples is that the first one composes the operations using 
multiple generators (separated by comma) and the second one uses parallel comprehensions (separated
by bar).

When you run the first example, the program prints meow, meow, woof, woof, woof and then returns 
a string ``cat and dog''. The second program interleaves the steps of the two computations and 
prints meow, woof, meow, woof, woof and then returns the same string.	

% ==================================================================================================

\section{Composing computations in parallel}

In the previous section, we used the parallel comprehension syntax to create computations that 
model parallelism using resumptions. Resumptions can be viewed as lightweight cooperative threads.
They are useful abstraction, but the simple implementation in the previous section does not give
us any speed-up on multi-core CPU. This section will follow a similar approach, but we look at how 
to implement actual parallelism based on \textit{evaluation strategies}. 

Marlow et al. \cite{strategies-new} introduced an \ident{Eval} monad for explicitly 
specifying the evaluation order. I start the section by briefly introducing the monad, so you don't 
need to worry if you're not familiar with it already. Then I'll demonstrate how to define a 
\ident{MonadZip} instance for this monad. This way, we can use the parallel 
comprehension syntax for running computations actually in parallel.

% --------------------------------------------------------------------------------------------------

\subsection{Introducing evaluation-order monad}

The evaluation-order monad is represented by a type |Eval a|. When writing code inside the monad,
we can use several functions of type |a -> Eval a| that are called \textit{strategies}. These
functions take a value of type |a|, which may be unevaluated, and wrap it inside the monad.
A strategy can specify how to evaluate the (unevaluated) value. The two most common strategies
are |rpar| and |rseq| (both are functions of type |a -> Eval a|). The |rpar| strategy starts 
evaluating the value in background and |rseq| evaluates the value eagerly before returning. 

A typical pattern is to use the do notation to spawn one computation in parallel and then 
run another computation sequentially. This way we can easily parallelize two function calls:

\begin{verbatim}
fib38 = runEval $ do 
  a <- rpar $ fib 36
  b <- rseq $ fib 37
  return $ a + b
\end{verbatim}
The example shows how to calculate 38$^\text{th}$ Fibonacci number. It starts calculating 
|fib 36| in parallel with the rest of the continuation and then calculates |fib 37| sequentially.
The do notation creates a value of type |Eval Integer|. We then pass the value to |runEval|, 
which returns the wrapped value. Because we used the |rpar| and |rseq| combinators when constructing 
the computation, the returned value will be already evaluated.

However, it is worth noting that the |return| operation of the monad doesn't specify any evaluation 
order. The function |runEval . return| is just an identity function that doesn't force evaluation 
of the argument. The evaluation order is specified by additional combinators such as |rpar|.

The \ident{Eval} monad is implemented in the \ident{parallel} package \cite{parallelpkg} and very 
well explained in the paper by Marlow et al. \cite{strategies-new}. You can find the definition of 
|Eval a| type and it's monad instance in Figure~\ref{fig:eval-monad}. We don't need to know how 
|rpar| and |rseq| work, so we omit them in the listing. 

\begin{figure}
\begin{verbatim}
data Eval a = Done a

runEval :: Eval a -> a
runEval (Done x) = x

instance Monad Eval where
  return x = Done x
  Done x >>= k = k x
\end{verbatim}
\caption{ Evaluation-order monad \ident{Eval} with a \ident{Monad} instance }
\label{fig:eval-monad}
\end{figure}

The |Eval a| type is simple. It just wraps a value of type |a|. The |runEval| function unwraps 
the value and the \ident{Monad} instance implements composition of computations in the 
usual way. The power of the monad comes from the annotations we can add. When using the monad, 
we're transforming values of type |a| into values of type |Eval a|. When doing so, we can specify 
the evaluation strategy. The strategy is usually given using combinators, but if we add an 
instance of the \ident{MonadZip} class, we can also specify the evaluation order using the monad 
comprehension syntax.

% --------------------------------------------------------------------------------------------------

\subsection{Specifying parallel evaluation order}

The |mzip| operator for the evaluation order monad encapsulates a common pattern that runs two
computations in parallel. We've seen an example in the previous section -- the first 
computation is started using |rpar| and the second one is evaluated eagerly in parallel using |rseq|.
You can find the implementation in Figure~\ref{fig:eval-zip}.

\begin{figure}
\begin{verbatim}
instance MonadZip Eval where
  mzip ea eb = do
    a <- rpar $ runEval ea
    b <- rseq $ runEval eb
    return (a, b)
\end{verbatim}
\caption{Instance of \ident{MonadZip} for parallelizing tasks}
\label{fig:eval-zip}
\end{figure}

A tricky aspect of |Eval| is that it may represent computations with explicitly
specified evaluation order (created, for example, using |rpar|). We can also create computations 
without specifying evaluation order using |return|. The fact that |return| doesn't evaluate the 
values makes it possible to implement the |mzip| function, because the required type signature is 
|Eval a -> Eval b -> Eval (a, b)|.

The two arguments already have to be values of type |Eval|, but we want to specify the evaluation
order using |mzip| after creating them. If all |Eval| values were already evaluated, then the |mzip|
operation couldn't have any effect. However, if we create values using |return|, we can then apply
|mzip| and specify how they should be evaluated later. This means that |mzip| only works for
|Eval| computations created using |return|.

Once we understand this, implementing |mzip| is quite simple. It extracts the underlying 
(unevaluated) values using |runEval|, specifies the evaluation order using |rpar| and |rseq| and 
then return a result of type |Eval (a, b)|, which now carries the evaluation order specification. 
Let's look how we can write a sequential and parallel version of a snippet that calculates
38$^\text{th}$ Fibonacci number:

\begin{verbatim}
fibTask n = return $ fib n

fib38seq = runEval [ a + b | a <- fibTask 36
                           , b <- fibTask 37 ]
fib38par = runEval [ a + b | a <- fibTask 36
                           | b <- fibTask 37 ]
\end{verbatim}

The snippet first declares a helper function |fibTask| that creates a delayed value using 
the sequential |fib| function and wraps it inside the |Eval| monad without specifying evaluation
strategy. Then we can use the function as a source for generators in the monad comprehension
syntax. The first example runs the entire computation sequentially -- aside from some wrapping
and unwrapping, there are no evaluation order specifications. The second example runs the 
two sub-computations in parallel. The evaluation order annotations are added by the |mzip| 
function from the desugared \textit{parallel comprehension} syntax.

To run the program using multiple threads, you need to compile it using GHC with the |-threaded|
option. Then you can run the resulting application with command line arguments |+RTS -N2 -RTS|,
which specifies that the runtime should use two threads. I measured the performance on a dual-core 
Intel Core 2 Duo CPU (2.26GHz). The time needed to run the first version was approximately 13 
seconds while the second version completes in 9 seconds.

\subsection{Writing parallel algorithms}
The ~1.4x speedup is less than the maximal 2x speedup, because the example parallelizes two 
calculations that do not take equally long time. To generate a better potential for parallelism,
we can implement recursive |pfib| that splits the computation into two parallel branches 
recursively until it reaches some threshold:

\begin{verbatim}
pfib :: Integer -> Eval Integer
pfib n | n <= 35 = return $ fib n
pfib n = [ a + b | a <- pfib $ n - 1 
                 | b <- pfib $ n - 2 ]
\end{verbatim}
I hope you'll agree that the declaration looks quite neat. A nice consequence of using parallel 
comprehensions is that we can see which parts of the computation will run in parallel without any 
syntactic noise. We just replace a comma with a bar to get a parallel version! The compiler also 
prevents us from trying to parallelize code that cannot run in parallel, because of data 
dependencies. For example, let's look at the Ackermann function: 

\begin{verbatim}
ack :: Integer -> Integer -> Eval Integer
ack 0 n = return $ n + 1
ack m 0 = ack (m - 1) 1
ack m n = [ a | na <- ack m (n - 1)
              , a <- ack (m - 1) na ]
\end{verbatim}
The Ackermann function is a well-known function from computability theory. It is interesting, 
because it grows really fast (as a result, it cannot be expressed using primitive form of 
recursion). For example, the value of |ack 4 2| is $2^{65536} - 3$. 

We're probably not going to be able to finish the calculation, no matter how many cores our CPU 
has. However, we can still try to parallelize the function by replacing the two sequential 
generators with parallel comprehension:

\begin{verbatim}
ack m n = [ a | na <- ack m (n - 1)
              | a <- ack (m - 1) na ]
\end{verbatim}
If you try compiling this snippet, you get an error message saying ``Not in scope: na''. We can 
easily see what went wrong if we look at the desugared version:

\begin{verbatim}
(ack m (n - 1)) `mzip` (ack (m - 1) na) >>= (\(na, a) -> a)
\end{verbatim}
The problem is that the second argument to |mzip| attempts to access the value |na| which is 
defined later. The value is the result of the first expression, so we can access it only after 
both of the two parallelized operations complete. 

In other words, there is a data dependency between the computations that we're trying to 
parallelize. If we were not using parallel monad comprehensions, we could mistakenly think 
that we can parallelize the function and write the following:

\begin{verbatim}
ack m n = runEval $ do 
  na <- rpar $ ack m (n - 1)
  a <- rseq $ ack (m - 1) na
  return a
\end{verbatim}
This would compile, but it wouldn't run in parallel! The value |na| needs to be evaluated before 
the second call, so the second call to |ack| will block until the first one completes. This 
demonstrates a nice aspect of writing parallel computations using the comprehension syntax. 
Not only that the syntax is elegant, but the desugaring also performs a simple sanity check on our code.

% ==================================================================================================

\section{Parallel comprehension laws}
I intentionally postponed the discussion about laws to the end of the article. So far, we looked
at three different implementations of the |mzip| operation. I discussed some of the expectations
informally to aid the intuition about the operation. The type of |mzip| partially specifies how 
the operation should behave, but not all well typed implementations are intuitively right.

To my understanding, the laws about |mzip| are still subject to discussion, although some were 
already proposed in the discussion about the GHC patch \cite{bringbackmc}. I hope to contribute to
the discussion in this section. We first look at the laws that can be motivated by category theory
behind the operation and then discuss additional laws inspired by the work on 
\fsharp \ joinads.

% --------------------------------------------------------------------------------------------------

\subsection{Basic laws of parallel bindings}
As already briefly mentioned, the |mzip| operation can be viewed as a \textit{natural 
transformation} defined by some \textit{monoidal functor} \cite{monoidalfunctor}. In practice,
this means that the |mzip| operation should obey two laws. The first one is usually called 
\textit{naturality} and it specifies the behavior of |mzip| with respect to the |map| function of 
a monad (the function can be implemented in terms of bind and return and corresponds to 
|liftM| from Haskell base library). The second law is \textit{associativity} and we can express it 
using a helper function $assoc((a,b),c)=(a,(b,c))$:

\begin{equation}
  \text{map} \: (f \times g) \: (\text{mzip} \: a \: b) \equiv \text{mzip} \: (\text{map} \: f \: a) \: (\text{map} \: g \: b)
  \label{law:naturality}
\end{equation}
\begin{equation}
  \text{mzip} \: a \: (\text{mzip} \: b \: c) \equiv \text{map} \: \text{assoc} \: (\text{mzip} \: (\text{mzip} \: a \: b) \: c) 
  \label{law:assoc}
\end{equation}
The naturality law (\ref{law:naturality}) specifies that we can change the order of applying 
|mzip| and |map|. The equation has been already identified as a law in the discussion about the
patch \cite{bringbackmc}. The law is also required by \textit{applicative functors} 
\cite{applicative} -- this is not surprising as applicative functors are also based on monoidal 
functors from category theory. This is quite common law that is also required by 
\textit{applicative functors}.

\UndefineShortVerb{\|}
\DefineShortVerb{\_}

The \textit{associativity} law (\ref{law:assoc}) is also very desirable. When we write a 
comprehension such as _[e | a <- m1 | b <- m2 | c <- m3]_, the desugaring first needs to zip some
of the three inputs and then add the third one, because _mzip_ is a binary operation. The order
of zipping feels like an implementation detail, but if we don't require associativity, it may 
affect the meaning of our code. This can lead to a very unexpected behavior.

In category theory, a monoidal functor also comes with a special value called 
\textit{unit}. If we combine unit with any other value using _mzip_, we can recover the original
value using _map snd_. Every applicative functor in Haskell defines a unit. For example,
the unit value for \ident{ZipList} is an infinite list of _()_ values. If we similarly extended 
\ident{MonadZip} with _munit_, the expression _[ e | a <- m ]_ would be equivalent to
_[ e | a <- m | () <- munit]_. It is important to realize that the computation created using monadic 
_return_ isn't the same thing as unit! For lists, _return ()_ creates a singleton list. The 
_return_ operation is unit of a monoidal functor defined by the monad, but the unit associated 
with |mzip| belongs to a different monoidal functor! 

% --------------------------------------------------------------------------------------------------

\subsection{Symmetry of parallel binding}

Another sensible requirement for _mzip_ (which exists in \fsharp \ joinads) is that reordering of 
the arguments only changes the order of elements in the resulting tuple. In theory, this means that 
the \textit{monoidal functor} defining _mzip_ is \textit{symmetric}. We can specify the law using a 
helper function $swap(a,b)=(b,a)$:

\begin{equation}
  \text{mzip} \: a \: b \equiv \text{map} \: \text{swap} \: (\text{mzip} \: a \: b)
  \label{law:sym}
\end{equation}
The \textit{symmetry} law (\ref{law:sym}) specifies that _[(x, y) | x <- a | y <- b]_ should mean 
the same thing as _[(x, y) | y <- b | x <- a]_. This may look like a very strong requirement, but 
it fits quite well with the usual intuition about the _zip_ operation and parallel monad 
comprehensions. The symmetry law holds for lists, parsers and the evaluation order monad. For poor 
man's concurrency monad, it holds if we treat effects that occur within a single step of the 
evaluation as unordered (which may be a reasonable interpretation). For some monads, such as the
_State_ monad, it is not possible to define a symmetric _mzip_ operation. 

\UndefineShortVerb{\_}
\DefineShortVerb{\|}

The three laws that we've seen so far are motivated by the category theory laws about 
\textit{monoidal functors}, or \textit{symmetric monoidal functor} that defines our |mzip| 
operation. However, we also need to relate the functor in some way to the monad that we're 
combining it with.

% --------------------------------------------------------------------------------------------------

\subsection{Relations with additional operations}

The discussion about the patch \cite{bringbackmc} suggests one more law that relates the |mzip|
operation with the |map| operation of the monad. It is called \textit{information preservation}
and it looks as follows:

\begin{equation}
  \text{map} \: \text{fst} \: (\text{mzip} \: a \: b) \equiv a \equiv \text{map} \: \text{snd} \: (\text{mzip} \: b \: a)
  \label{law:ipreserve}
\end{equation}

The law specifies that combining a computation with some other computation using |mzip| and then 
recovering the original form of the value using |map| doesn't lose information. This laws is a bit 
tricky though. For example, it doesn't hold for lists if $a$ and $b$ are lists of different length. 
The |zip| function restricts the length of the result to the length of the shorter list. Similarly, it 
doesn't hold for parsers (from the first section) that consume different number of characters. 

However, the law expresses an important requirement -- when we combine certain computations, it should be
possible to recover the original components. A similar law holds about applicative functors, but with 
a slight twist. Instead of zipping two arbitrary monadic values, the law zips an arbitrary value with
\textit{unit}. For lists, unit is an infinite list, so the law holds.

Ideally, we'd like to say that combining two values with the same structure creates a new value
which also has the same structure. There is no way to refer to the ``structure'' of a monadic value, 
but we can create such values using the |map| function. This weaker law holds for both lists and 
parsers. Additionally, we can also describe the case when one of the two computations is |mzero| and
when both of them are created using |return|:

\begin{equation}
  \text{map} \: \text{fst} \: (\text{mzip} \: a \: (\text{map} f a)) \equiv a \equiv \text{map} \: \text{snd} \: (\text{mzip}\: (\text{map} \: g \: a) \: a)
  \label{law:weakipres}
\end{equation}
\begin{equation}
  \text{map} \: \text{fst} \: (\text{mzip} \: a \: \text{mzero}) \equiv mzero \equiv \text{map} \: \text{snd} \: (\text{mzip} \: \text{mzero} \: a)
  \label{law:zero}
\end{equation}
\begin{equation}
  \text{mzip} \: (\text{return} \: a) \: (\text{return} \: b) \equiv \text{return} \: (a, b)
  \label{law:unitmerge2}
\end{equation}

The first law (\ref{law:weakipres}) is quite similar to the original law (\ref{law:ipreserve}).
The only difference is that instead of zipping with an arbitrary monadic value $b$, we're zipping $a$ 
with a value constructed using |map| (for any complete functions $f$ and $g$). This means that the 
actual value(s) that the monadic value contains (or produces) can be different, but the structure will 
be the same, because well-behaved |map| preserves the structure. 

The second law (\ref{law:zero}) specifies that |mzero| is the \textit{zero element} with respect to 
|mzip|. It looks almost necessary, because |mzero| with type |m a| doesn't contain any value of 
type |a|, so there is no way we could construct a value of type |(a, b)|. The law complements the 
first law (\ref{law:weakipres}), but it is not difficult to see that it would contradict with the 
original \textit{information preservation} (\ref{law:ipreserve}).

\UndefineShortVerb{\|}
\DefineShortVerb{\_}

Finally, the third law (\ref{law:unitmerge2}) describes how _mzip_ works with respect to the monadic 
_return_ operation. This is interesting, because we're relating _mzip_ of one applicative functor with 
the unit of another applicative functor (defined by the monad). The law isn't completely arbitrary. 
An equation with a similar structure is required for \textit{causal commutative arrows} \cite{causalarr}.
In terms of monad comprehension syntax, the law says that _[ e | a <- return e1 | b <- return e2 ]_ 
is equivalent to _[ e | (a, b) <- return (e1, e2)]_.

\UndefineShortVerb{\_}
\DefineShortVerb{\|}

I believe that the three laws I proposed in this section partly answer the question how to relate
the two structures combined by parallel monad comprehensions -- the \ident{MonadPlus} type class
with the symmetric monoidal functor that defines |mzip|.

% ==================================================================================================

\section{Conclusions}
After some time, monad comprehensions are back in Haskell! The recent GHC patch makes them even 
more useful by generalizing additional features of list comprehensions including parallel binding 
and the support for operations like ordering and grouping. In this article, I focused on the first 
generalization, but the remaining two are equally interesting.

We looked at three examples: parallel composition of parsers (which applies 
parsers on the same input), parallel composition of monads for cooperative concurrency (which interleaves
the steps of computations) and a parallel composition of evaluation order monads (which runs two 
computations in parallel). Some of the examples are inspired by my previous work on joinads that add 
similar language extension to \fsharp. The \fsharp \ version includes an operation very similar to 
|mzip| from the newly included \ident{MonadZip} type class. I also proposed several laws -- some of 
them inspired by category theory and some of them inspired by my work on \fsharp \ joinads -- hoping 
that this article may contribute to the discussion about the laws required by \ident{MonadZip}.

\subsection{Acknowledgements}
Thanks to Alan Mycroft for inspiring discussion about some of the monads demonstrated in this article
and to Dominic Orchard for many useful comments on a draft of the article as well as discussion
about category theory and the |mzip| laws.

\bibliography{CompreFun}

\end{document}
























